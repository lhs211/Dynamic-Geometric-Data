\documentclass{ecmm427_assignment}
\usepackage{blindtext} % package just required to create the filler text
\usepackage{url} % required as the bibfile has a url in it
\usepackage{cite}
\usepackage{graphicx}
\usepackage{hyperref}
\usepackage{amsmath,amsfonts,amssymb}
\usepackage{array,multirow}
\usepackage[labelfont=bf]{caption}
\usepackage{subcaption}
\usepackage[export]{adjustbox}
\usepackage{float}
\usepackage{tikz}
\usetikzlibrary{positioning}

\begin{document}
\title{Data Structure for Dynamic Geometric Data}
\author{086696}
\maketitle

\begin{abstract}
\centering
There are a range of data structures available for the storage and efficient update of unconstrained sets of mutually non-dominating solutions. However there is little attention given to structures which are unbound, non-elitist and able to manage dynamic data, meaning a solution's objectives are re-evaluated over time. Objectives may be re-evaluated due to dynamic changes in the problem itself or due to estimates of objectives changing in noisy problems. A framework is proposed and an existing data structure is implemented for managing these relations between such dynamic solutions.
\end{abstract}

\null
\vfill
\declaration

\newpage % forcing a new page to separate the body of the report from the coverpage
\tableofcontents
\newpage

\section{Introduction}
A multi-objective optimization problem (MOOP) can be stated as a vector of decision variables which attempts to optimize a vector of objective functions while satisfying constraints \cite{Altwaijry2012}. The answer to this problem is the set of solutions that define the optimal trade-off between the competing objectives, which is referred to as the \textsl{Pareto-optimal front}.\\
Given two objective vectors, \textbf{v} and \textbf{u}, \textbf{u} is said to \textsl{dominate} \textbf{v} iff $\textbf{u}_d$ $\leq$ $\textbf{v}_d$, $\forall$d = 1,...,$D$ and \textbf{u} $\neq$ \textbf{v}. This is denoted as \textbf{u}$\prec$\textbf{v}. A feasible solution lying on this Pareto-optimal front is non-dominated as it cannot improve upon any objective without degrading at least one of the others therefore given the constraints of the model, it follows that no solutions may exist beyond this true Pareto-optimal front. \\
The set of solutions which make up the Pareto-optimal front are \textsl{mutually non-dominating} \cite{Fieldsend2014} as demonstrated in Figure \ref{fig:pareto}. Thus the goal of multi-objective algorithms is to locate the Pareto front that consists of these non-dominated solutions, but to also find a set of non-dominated solutions as diverse as possible so that the entire range of the Pareto-optimal front is represented \cite{Fieldsend2003} \cite{Deb2011}.

%pareto front figure%
\begin{figure}[h]
    \centering
    \includegraphics[scale=0.5]{pareto-front}
    \caption{Pareto front}
    \label{fig:pareto}
\end{figure}

MOOPs have many real world applications across numerous disciplines. They are employed when there is a trade off between the best set of solutions, whether this be design parameters for an engineering problem or to aid in designing a neural network for an existing problem \cite{Stewart2008}.\\
Evolutionary Algorithms (EAs) are a successful and widely used general problem solving method that mimic, in a simplified manner, biological evolution. In nature, individuals have to adapt to their environment in order to survive in a process called \texttt{evolution}, where features which make an individual more suited to compete are preserved when it reproduces, and those features which make it weaker are eliminated. Over subsequent generations the fittest individuals survive, with their best features being passed on to their descendants during the sexual recombination process known as crossover \cite{Coello2005}. As the number of generations increases, the overall fitness of the population should also increase.\\
Multi-objective evolutionary algorithms (MOEAs) maintain a population of solutions, while dealing simultaneously with a set of potential solutions as they try to converge towards the Pareto-optimal front. It is common in evolutionary multi-objective optimization to keep track of the non-dominated individuals by storing them in a special data structure called an archive \cite{Yuen2012}. An archive \texttt{A} of non-dominated solutions is a data structure for keeping track of the known non-dominated points $\{$$x^{(1)}$$, . . . ,$$x^{(n)}$$\}$ belonging to \texttt{X} of a MOOP \cite{Glasmachers2017}.\\
The archive is updated during the optimization process whenever an individual in the population changes. The computational cost of updating the archive is critical \cite{Altwaijry2012} \cite{Drozdik2015}.\\
The aim of this project is to implement the guardian dominator approach \cite{Fieldsend2014} which stores the dominance relations between vectors in an unbounded dynamic non-elitist archive. The key challenges faced by this approach is maintaining the set of dominance relations whenever a new solution is added, or when an existing solutions objective values change, due to the problem being dynamic in nature.

\subsection{Literature Review Summary}
\label{literature}
The existing literature reviewed as part of this project looked into a range of data structures used to represent such archives. It also investigated the dynamic, noisy and robustness qualities of data.

\subsubsection{Elitist MOEAs}
Elitist MOEAs use external archives to store the non-dominated solutions of each generation, thus the archive contains the best estimate of the Pareto-optimal front at any stage \cite{Altwaijry2012} \cite{Everson2002}. The final output of an elitist MOEA is the set of mutually non-dominated solutions stored in the final archive, however this set remains an approximation of the Pareto-set \cite{Mostaghim2005}. A non-elitist algorithm will also store dominated solutions as part of its archive. An archive is considered bounded if the number of points it can store is limited. This is a frequently used approach allowing reduction of the time needed to update a Pareto archive \cite{Yuen2012}.

\subsubsection{Bounded Archives}
In most MOEAs, the truncation procedure on an archive is used to maintain the diversity of solutions, preserving the boundary individuals and deleting the individuals with the highest density. Having too many non-dominated solutions might reduce selection pressure and slow down the search \cite{Li2008}. It is called \textsl{overnondomination} when a large proportion of the population becomes non-dominated. This is considered disadvantageous \cite{Drozdik2015}. However truncation leads to retreating/oscillating, deteriorating quality, and slowing the rate of convergence, as it can potentially discard non-dominated points. 
\\Forming an archive of all non-dominated solutions will mean, that the archive always moves towards the true Pareto front as it cannot retreat or oscillate, allowing for sensible convergence criteria to be defined. When the frontal set contains all the currently found non-dominated objective vectors, it will become very large as the search progresses, and as it has to be queried at each generation, it can quickly become costly to maintain \cite{Everson2002} \cite{Schutze2003}. A bounded archive will always have lesser quality than an unbounded archive, thus if an unbounded archive can be efficiently managed and updated, it is advantageous to use \cite{Jaszkiewicz2016}.

\subsubsection{DMOOPS}
Optimisation problems with at least two objectives in conflict with one another, and where either the objectives or constraints (or both) change over time, are referred to as dynamic multi-objective optimisation problems (DMOOPs). Solving DMOOPS requires that the Pareto-optimal set at different moments can be accurately found \cite{Jiang2018}.
\\The goal should be to track a solutions movement through the solution and time spaces as closely as possible. The building blocks for solving dynamic optimisation problems are: maintenance of diversity so that the population can adapt more easily to changes; react to changes such as performing a hypermutation; use of memory to help preserve good solutions from the past (only useful when what occurs now can happen in the future); and having multiple populations as a way of increasing diversity \cite{Cruz2011}.\\
At the start of the evolutionary algorithm run it is desired to determine both the dominated and non-dominated individuals, store these in the archive and then update the archives contents as the population changes \cite{Drozdik2015}. The population will update after each generation, meaning the Pareto-optimal set is likely to have changed, therefore the dominance relations between vectors will have also changed thus the archive will require an update. An update has two key operations: insert, in which a new solution is compared to the existing contents of the archive to check if it is non-dominated \cite{Altwaijry2012}; and edit, in which a vector has moved position in the archive, either becoming dominated or non-dominated.

\subsubsection{Linear List Approach}
In most MOEAs with elitism, the population is stored within a linear list due to its simplicity in implementation and use. In this structure, a new point is compared to all points in the list until a covering point is found or all points are checked. The point is only added if it is non-dominated with respect to all points in the list. In the worst case, the whole list is browsed before adding a point. Any newly non-dominated points are appended to the end of the list. The complexity, in terms of the number of points comparison, is thus $O(N)$ where $N$ is the size of the list \cite{Yuen2012} \cite{Glasmachers2017}  \cite{Jaszkiewicz2016} \cite{Shi2009}.\\
The basic linear list approach can be improved upon by reducing the number of dominance comparisons by trying to infer domination relationships using the transitive nature of Pareto dominance. Relations can be deduced from existing relations, such that if \texttt{A} dominates \texttt{B} and \texttt{B} dominates \texttt{C} then we can deduce that \texttt{A} also dominates \texttt{C}. These algorithms achieve a very significant speedup, but they are specifically designed for populations where the domination relationship between individuals is relatively common. This assumption does not hold for problems with a large number of objectives. The fewer domination relationships there are, the fewer such relationships can be inferred, and in turn, the performance suffers. These methods do constitute a significant innovation since they are dynamic in that when one individual changes, the information about the non-dominated fronts can be efficiently updated \cite{Fieldsend2014} \cite{Drozdik2015} \cite{Shi2009}.

\subsubsection{Guardian Dominator Approach}
The proposed data structure is based on the guardian dominator approach \cite{Fieldsend2014}, which employs previously reviewed techniques to store the domination relations between vectors in an unbounded dynamic non-elitist archive. The main idea is that each member of the Pareto-set becomes a guardian of the solution it dominates in an effort to prevent keeping track of all domination relations between members of said archive. Attempts are made to store the minimum number of links to appropriately place new solutions into the archive. How a guardian is assigned is dependent on the parameters given to the algorithm. Some options only store which elements the Pareto-set dominates whereas others attempt to create chains between dominated solutions. This data structure is expected to perform especially well when objective values rapidly change as it allows for quick updates.

\subsection{Project Specification Summary}
This project's aim is to develop a package in Java which will implement a data structure suitable for playing the role of an unbounded dynamic non-elitist archive in a MOEA. It should be able to efficiently maintain the mutually non-dominating set when the objective values change over time, in that at any time step, the current estimate of the Pareto-optimal front can be retrieved.\\
The basis of the implementation of the data structure is detailed in \cite{Fieldsend2014}, which describes an algorithm that uses the premise of guardian dominators to determine dominance relations between members of the archive.\\
The data structure described in the \cite{Fieldsend2014} already has an implementation in MATLAB, however a Java implementation has been specially highlighted to be useful to the community that deals with the type of problems where such a data structure would be beneficial. An object oriented approach using references between the guards and guarding items is envisioned to be quicker for larger populations.

\subsubsection{Non Functional Requirements}
The desirable, non functional attributes of the data structure describe aspects and features which are not essential to its operation, however will be beneficiary in making the data structure viable as part of a feature rich package. Therefore it follows the archive should be thread safe in order for it to function correctly during simultaneous execution, allowing multiple threads to access the same shared data, but only allowing one thread to modify its content at any given time. It was originally outlined in the previous project specification that the data structure should be multi-threaded, however this was very much an oversight as the actions of adding and editing solutions are inherently atomic thus the only place multi-threading could be employed is when multiple solutions are requested to be operated on in a batch.
\\Dynamic, noisy, and robust data should all be able to be suitably managed by the archive in order to improve the range of problem types which the archive is suitable for. This can be done by detecting duplicates when adding and editing solutions.
\\It was also desired that the Pareto-optimal set be diverse. This will be accomplished by the archive being unbounded and non elitist as the absence of truncation means no solutions will be discarded. If not solutions are discarded, the diversity of the elite set will not be reduced.
\\In addition to the concrete implementation developed, an interface has been proposed for such a  data structure, as this will allow others to use, build on, and improve the provided implementation. An interface allows for the accurate description of the role the data structure will play, and what use cases it will be able to handle. It will also allow a set of unit tests to be created, which all future implementations must conform under, to ensure the correct behaviour is performed under usage scenarios.

\subsubsection{Evaluation}
An empirical analysis of the data structure will be performed to evaluate the performance characteristics under a range of varying use types and protocols, in order to identify how best to configure the parameters of the archive, as well as how best to investigate new extensions or current bottlenecks of the concrete  implementation. Particular focus will be given to the method of guardian assignment at each stage. The overarching aim of this project is to develop an implementation of a data structure that should offer significant improvements in performance and efficiency over the standard linear list approach for handling when objective values change in a MOOP. This is the hypothesis that is to be tested, and in order for this to be considered proved, the following set of functional requirements should be met with sufficient performance.

\begin{enumerate}
    \itemsep-0mm
    \item Archive is unbounded.
    \item Archive is non-elitist.
    \item Be able to evaluate new locations.
    \item Be able to change the objective values of a dominated member of the archive.
    \item Be able to change the objective values of a non-dominated member of the archive.
    \item Be able to retrieve the non-dominated set at the current time step.
    \item Perform in better runtime than a linear list approach.
\end{enumerate}

To achieve these criteria, the project can be split into three stages: design and planning, development, and analysis of results. Once these phases are completed the criteria can be evaluated. If all the evaluation criteria are met, then the hypothesis can be considered proved.

\section{Design}
The first stage in achieving the evaluation criteria was to create a design of the system which could be used as a blue print to follow throughout the development of the archive. Correct thorough planning would allow for the evaluation criteria to stay at the core of the project, averting the risk of losing scope of the objectives.

\subsection{API}
The design process began by deciding what would be offered in the public application programming interface (API). This would determine what features would be offered to end users of the package. The creation of an interface will allow for multiple implementations of the archive to be developed in future. This also means that one set of unit tests can be published against the interface, but with multiple implementations being able to be tested from them, thus reducing development time of future implementations and increasing consistency amongst implementations.\\
The interface supplied provides a general framework for dynamic unbounded non-elitist archives of which a \texttt{Guardian Archive} has been implemented as part of this paper. A dynamic solution abstract class is provided which contains a property listener attached to the objective values, which will trigger an archive update upon any changes.\\
The key methods in the interface are \texttt{add} and \texttt{editObjectives}. These will be called when an update to the archive will need to be triggered as there are new objective values. There is the option of adding multiple solutions at once, however this will treat each addition as a single atomic action, sequentially adding them to the population. Further functionality of the API is checking that the population contains either a single solution, or a collection of solutions, regardless of whether or not the solution is non-dominated. There is also the functionality to retrieve the current non-dominated set as an iterable, and to check whether an individual solution is a member of the non-dominated set. The size of the Pareto set and the size of the entire population can also be retrieved. The API does not offer the functionality to delete solutions from the population, nor does it offer functionality to forcefully add solutions to the Pareto set without any checks as this would disrupt the internal logic of dominance relations between solutions.
\\Once the public \texttt{DynamicArchive} API was created, the \texttt{GuardianArchive} was implemented. It is parameterized to contain \texttt{GuardianSolution} or any of its sub-classes. \texttt{GuardianSolutions} are a sub class of \texttt{DynamicSolutions}, and they are employed to store the direct relations between individual objective vectors. They offer the package private functionality of each solution being allowed one parent solution (guardian dominator); and a \texttt{List} of child solutions. As new solutions are added, or the objective values of current solutions change, the child-parent relations between solutions will be updated as the relations represent the guardian dominance between solutions. This births the tree structure shown in Figure \ref{fig:tree}.

\begin{figure}[h]
    \centering
    \includegraphics[scale=0.5]{tree}
    \caption{Tree Structure}
    \label{fig:tree}
\end{figure}

\subsection{MOEA Integration}
The existing Java library, the MOEA framework \cite{}, has been extended in order to increase use-ability and accessibility of the package to developers. This has been done by extending the existing \texttt{Solution} class so that data will conform to the same standard used by existing archive implementations. Their Pareto dominance comparator has been used to check the dominance relations between two solutions as this is, tested and optimised.

\subsection{Data Representation}
The data will be stored in the class \texttt{GuardianSolution} which extends the abstract class \texttt{DynamicSolution}. This in turn extends the \texttt{Solution} class found in the MOEA framework. The $Solution$ class stores the variables, objectives, constraints, and any attributes the solution may have. The objective vector is stored as a primitive array. If a user wishes to create a new solution, they can  do so by either passing an existing solution to a \texttt{GuardianSolution} constructor or by initialising a \texttt{GuardianSolution} with a one-dimensional array of doubles. The \texttt{DynamicSolution} class builds on this by providing the framework for triggering an update to the archive through the use of a property change listener which watches the \texttt{objectives} field for any changes.
\\The \texttt{GuardianSolution} instances are encapsulated by the \texttt{GuardianArchive} class which is an implementation of a \texttt{DynamicArchive}. 
\\The list of solutions which make up the non-dominated set are also the roots to each tree in the forest, thus the archive need only store these solutions as the remainder of the archive can be inferred from the guardian relations. Every dominated solution must have a guardian, therefore is in turn a member of a tree.
\\The \texttt{GuardianArchive} class manages the relations between the solutions, ensuring that every parent solution dominates all of its children. By storing solutions in a tree structure, it allows the search space to be reduced when comparing dominance, as any solution stored in a level lower in the tree than a dominated solution will also be dominated, due to the transitive nature of dominance as described in the \nameref{literature}. A level order traversal of the tree has been used to search for solutions, because likely guardian candidates will be closer to the root as the population evolves, thus reducing the search space and in turn the number of domination comparisons.

%UML diagram of package%
\begin{figure}[h!]
    \centering
    \includegraphics[scale=0.6]{projectUML}
    \caption{UML}
    \label{fig:Class Diagram}
\end{figure}

\subsection{Performance}
The project requirements specify sufficient performance which is acknowledged to be ambiguous here. There is no baseline implementation to draw a comparison to as there has been very little research into unconstrained non-elitist dynamic archives. It is expected there will be worse performance compared to an elitist archive due to the larger population size, and there not being an ordering of Pareto solutions as is common in elitist archives.
\\To maximise performance, the implementation takes advantage of caching to reduce computation throughout. For loops are primarily indexed based to avoid the overhead of using an iterator. No maximum or minimum memory requirements were specified, as this will very much depend on the end users system architecture, therefore we do not want to preemptively limit performance by reducing the memory footprint at the expense of computation.

\subsection{Optimisation}
Naturally, we want to minimise the number of domination comparisons required at each time step therefore it follows that no two solutions should be compared more than once per add or edit operation. To achieve this, when comparing a solution against the current elite set, the index is recorded of the first member to dominate the newly added or edited solution. This index is only revisited if the solution requires tutelage from the elite set as no other guardian is available.
\\The archive favours guardians which are deeper in the tree as this will reduce the search space of future operations due to the transitive nature of dominance. For example, when assigning a guardian based on which dominating solution guards the fewest from a set, the elite set will only be searched for a guardian if there has not been a dominating member from the set so far, with the count of how many solutions a candidate guards, being reset for each component that makes up that set. This selection technique was chosen as there may be a member of the elite set which guards fewer solutions, but the solution which guards the fewest from a tree is a better candidate as it allows for the future search space to be reduced, owing to the level order search of a tree having the stopping condition not to continue if solutions are no longer dominating. This relation also holds true for when a guardian is assigned based on closest proximity.
\\The Euclidean distance has been used here to determine proximity however other metrics such as the Manhattan distance may be more appropriate, depending on the nature of the problem.

\subsection{Error and Exception Handling}
Methods return a boolean value based on if the operation was successful. The solutions have been sanitised for either being null or having an incorrect number of objectives.
The constructor of the \texttt{GuardianArchive} takes a \texttt{Duplicate Mode} enum as a parameter, which can configure the archive to either: \texttt{ALLOW\_DUPLICATES}, \texttt{NO\_DUPLICATE\_OBJECTIVES}, or \texttt{ALLOW\_DUPLICATE\_OBJECTIVES}. In the case of \texttt{ALLOW\_DUPLICATE\_OBJECTIVES}, it is the decision variables which are not allowed to be identical.
\\Scope is obeyed in that the user cannot access methods which affect the internal relations between nodes. These have been kept package private in order to protect the internal forest structure from external interference.

\section{Implementation}
Once the design was complete, it followed to begin the development process following a rigourous outlined methodology, which had been thought out for which technologies to use and how to adapt to changes in requirements.

\subsection{Methodology}
The nature of this project is primarily research based, as it investigates the benefits and drawbacks of using a guardian dominator archive implementation. The development methodology followed was an agile Kanban approach where the backlog of tasks was continuously updated as the requirements for each stage became more refined. The advantage of using an agile approach over the Waterfall approach is that the product requirements are able to adapt over time, which is of particular use when a problems complexity has not been fully understood \cite{Balaji2012}. My use of the methodology varied from this in that I would work towards a weekly deadline of my meeting with my supervisor, where I could review the previous weeks progress and refine the backlog of remaining tasks. My supervisor was in part acting as the product owner through the guidance they gave in regards to what features should be included and their format.

\subsection{Technology Stack}
The developement process began by setting up a development environment. IntelliJ was elected to be used for its rich range of features, such as its debugger and profiler.  The project uses the build tool gradle. The library JUnit 5 was chosen as the unit testing framework as it is the latest release, and the most popular. The version control system employed was Git, as is industry standard. The use of Git allowed for features to pushed to the main branch once code changes passed the unit tests. The remote repository can be found at %%github here%.

\subsection{Development}
Development work on the data structure began by writing a comprehensive set of unit tests against the API. This would ensure that all code changes made would not add any feature breaking bugs to the package. As the package grew in functionality, the code coverage of the tests unfortunately decreased as new scenarios were being ran which had not been properly accounted for by the test suite. This led to two helper test methods that access the package's internal dominance structure using reflection, being written. The methods \texttt{checkGuardianDominates}, and \texttt{writeDominationStructure} were especially beneficial as they conveyed when a dominated solution had not been assigned a parent, or when a cycle, in place of a tree, had been created. These were the two most common frustrating scenarios during the debugging process, occurring frequently when a solution had been misassigned a guardian.

\section{Empirical Analysis}
The empirical analysis performed aims to identify the performance characteristics of the data structure and how best to maximise these based on the problem type and guardian assignment technique.

\subsection{Experiments}
All empirical work was conducted on a laptop running Windows 10 with the machine specification:
Laptop Quad-core 2.8 GHz CPU. L1d cache: 128KB, L1i cache: 128KB, L2 cache 1MB 16GB Ram $@$ 2400MHz. It considered that the  performance of the data structure may vary with architecture, thus this is a limitation of the analysis provided. All timings have been recorded by measuring the CPU time dedicated to the execution thread when interacting with the data structure excluding all other time costs, such as sampling from the analytical distribution.\\
The first set of experiments use the protocol employed in \cite{Glasmachers2017} for generation of objective vectors from a controlled analytical distribution, which removes the stochastic element of an optimiser from the results. The archive is constructed from a sequence of \textit{D} dominated (d \textgreater\ 0) and N non-dominated (d=0) normally distributed objective vectors according to:

\begin{equation}
\label{eqn:Analytical Distribution}
    y^{(k)} \sim \mathcal{N} \left( \frac{d(N+D)}{k} \mathbf{1}, \mathbb{I} - \frac{1}{m} \mathbf{1} \mathbf{1} ^ T \right)
\end{equation}

where $\mathbb{I} \in \mathbb{R} ^ {m \times m}$ is the identity matrix and $\mathbf{1} = (1, 1, ..., 1) ^T \in \mathbb{R} ^ m$ is the vector of all ones. The distribution has unit variance in the subspace orthogonal to the $\mathbf{1}$ vector. The parameter d controls the systematic improvement of points. At position \textit{k} of the sequence it is assigned a value of 0 with a probability $c \frac{N'}{N' + D'}$, where $D$' and $N$' denote the number of remaining dominated and non-dominated points to be placed into the sequence, otherwise \textit{d} is assigned a value \textgreater\ 0. For $c$ \textless  1 there is a preference for seeing more dominated points later in the sequence, and for c \textgreater\ 1 there is a preference for seeing more dominated points earlier in the sequence. The non-zero value employed for $d$ is not specified in \cite{Glasmachers2017} hence we use $1/N$ here.\\
In the simulation, for each member of the population an underlying `true' objective location is stored. This `true' location is the mean as sampled from Equation \ref{eqn:Analytical Distribution}. An evaluation of a new solution is sampled from the mean at time $k$, giving the `observed' noisy objective vector. When a solution is reevaluated, another objective vector is drawn from the sample, and the $y$ of a solution is taken as the mean average of the noisy objective vector samples taken so far. This allows for objective vectors to have the refinement property as seen in noisy optimisation. In order for the desired values of N and D to be reached, each point must be evaluated numerous times before it reaches an approximation to its mean value, thus its' desired location. As points are selected randomly from either the dominated, or the non-dominated, set it is not guaranteed that points will converge to their mean objective values. Therefore, two approaches have been taken to sampling.
\begin{enumerate}
    \item The selected objective vector's $y$ is updated to its true location.
    \item The selected objective vector is sampled once, with its $y$ being the mean average of samples taken from that vector so far.
\end{enumerate}

The two iterative generating processes of a population being examined are the generation of new solutions according to Equation \ref{eqn:Analytical Distribution} where the number of non-dominated solutions is a parameter, and then selection of which member of the population to change:
\begin{enumerate}
    \item A random member of the population is changed.
    \item An elite member of the population is changed.
\end{enumerate}

This creates four distinct simulations when combining the sampling technique with a selection procedure. All four simulations have new solutions generated from the distribution that have been sampled once, hence they contain the desired multivariate isotropic Gaussian noise.
A solution is chosen using a selection method and its value is updated using a sampling method. This leads to the following simulations:

\begin{description}
    \item $S_1$: Noisily distributed new solutions, randomly changed true objective solutions.
    \item $S_2$: Noisily distributed new solutions, randomly changed true objective elite solutions.
    \item $S_3$: Noisily distributed new solutions, randomly changed converging solutions.
    \item $S_4$: Noisily distributed new solutions, randomly changed converging elite solutions.
\end{description}

Each simulation was run for the values of $c$ = \{0.5, 1.0, 1.5\}. The situation where there are lots of dominated points later in a sequence, then proportionally more non-dominated points later, is especially interesting as this will demonstrate how larger, longer chaining trees will behave.
Each simulation was also run for values of $m$ = \{2, 5, 10\} to analyse how the number of dimensions would affect the archive update procedure.
Finally, each simulation was run with a true elite set size of \{128, 512, 2048\}, as this will expose the tree structure's handling of updating the elite set.

\begin{table}[h]
    \centering
\begin{center}
    \begin{tabular}{| c | c | p{10cm} |}
    \hline
    Label & Option combinations & Description of guardian assignment \\ \hline
    C1 & 1.1, 2.1, 3.1, 4.1, 5.1, 6.1 & Assigns first dominating element to new solution, and first dominating to changed solution \\ \hline
    C2 & 1.2, 2.1, 3.1, 4.1, 5.1, 6.1 & Assigns closest elite to new solution and first dominating to changed solution \\ \hline
    C3 & 1.3, 2.1, 3.1, 4.1, 5.1, 6.1 & Assigns dominating elite which guards the fewest to new solution, and first dominating to changed solution \\ \hline
    C4 & 1.1, 2.2, 3.1, 4.2, 5.2, 6.2 & Assigns first dominating found - always searching through the changed sub tree first for the changed solution \\ \hline
    C5 & 1.2, 2.3, 3.2, 4.3, 5.3, 6.3 & Assign closest dominating solution from comparison sets \\ \hline
    C6 & 1.3, 2.4, 3.3, 4.4, 5.4, 6.4 & Assigns dominating solution with fewest guarded solutions from comparison sets \\
    \hline
    \end{tabular}
    \caption{\textbf{Methods combinations tested empirically. Options refer to Algorithms 1 and 2.}}
\label{table:combinations}
\end{center}
\end{table}

\subsection{Guardian Assignment Technique}
The configurations in Table \ref{table:combinations} were chosen as they cover a wide range of the methods described in \cite{Fieldsend2014}. Combinations $C_{1-3}$ vary only in their choice of which member of the elite set should become a guardian when a new solution is added. The remaining five options all give preference for choosing an immediate guardian without further comparison, if that's not an option, they give preference for solutions which are the first dominating members of the corresponding set.\\
Combinations $C_{4-6}$ take a consistent approach where the same assignment method is used at each option. This may not be considered optimal for all options, as sometimes the solution which is closest, or guards the fewest others, may also be the same solution that is chosen through the greedy option one approach, which takes far less comparisons as its search stops once a potential guardian is found, rather than searching for the highest quality guardian from each set.

%%%%%
% show how tree structure changes based on assignment technique?
%%%%%
\subsection{Empirical Results}
The simulations were ran for 100,000 time steps using the option configurations detailed in Table \ref{table:combinations}. The procedure followed is that, at alternate time steps, either a new solution is added to the population, or an existing member of the population has its objective vector changed. The time taken to perform the operation, the number of comparisons required, and the size of the elite archive were all recorded at each time step thus add and edit operations can be compared independently.\\

\subsubsection{Timings}
Figure \ref{fig:Timing_c=1.0} shows how a population with solutions spread evenly throughout its lifespan performs. The logarithmic scale being used shows how the cumulative time taken to update the archive increases as the archive becomes larger. The growth follows a more discrete pattern as there are a multitude of operations being performed, all of which are not equal in cost thus causing sudden spikes in growth.

%%%%%%%%%%%%%
% Timing
%   timing128_c=1.0
%   timing512_c=1.0
%   timing2048_c=1.0
%%%%%%%%%%%%%
\begin{figure}[h]
    \centering
    \begin{tikzpicture}[image/.style = {inner sep=0pt, outer sep=0pt}, node distance = 1mm and 1mm] 
% Timing 128, c=1.0
\node [image] (frame1)
    {\includegraphics[width=0.15\linewidth]{plots/population_timings_c1_D=2_NON_DOM=128_c=1.0.png}};
    \node[left = of frame1, node distance=0cm and 0cm, rotate=90, anchor=center] {\tiny{\textbf{Timings (ms)}}};
    \node[above = of frame1, node distance=0cm and 0cm] {$C_1$};
\node [image,right=of frame1] (frame2) 
    {\includegraphics[width=0.15\linewidth]{plots/population_timings_c2_D=2_NON_DOM=128_c=1.0.png}};
    \node[above = of frame2, node distance=0cm and 0cm] {$C_2$};
\node[image,right=of frame2] (frame3)
    {\includegraphics[width=0.15\linewidth]{plots/population_timings_c3_D=2_NON_DOM=128_c=1.0.png}};
    \node[above = of frame3, node distance=0cm and 0cm] {$C_3$};
\node[image,right=of frame3] (frame4)
    {\includegraphics[width=0.15\linewidth]{plots/population_timings_c4_D=2_NON_DOM=128_c=1.0.png}};
    \node[above = of frame4, node distance=0cm and 0cm] {$C_4$};
\node[image,right=of frame4] (frame5)
    {\includegraphics[width=0.15\linewidth]{plots/population_timings_c5_D=2_NON_DOM=128_c=1.0.png}};
    \node[above = of frame5, node distance=0cm and 0cm] {$C_5$};
\node[image,right=of frame5] (frame6)
    {\includegraphics[width=0.15\linewidth]{plots/population_timings_c6_D=2_NON_DOM=128_c=1.0.png}};
    \node[above = of frame6, node distance=0cm and 0cm] {$C_6$};

% Timing 512, c=1.0
\node [image] (frame11) [image,below = of frame1]
    {\includegraphics[width=0.15\linewidth]{plots/population_timings_c1_D=2_NON_DOM=512_c=1.0.png}};
    \node[left = of frame11, node distance=0cm and 0cm, rotate=90, anchor=center] {\tiny{\textbf{Timings (ms)}}};
\node [image,right=of frame11] (frame12) 
    {\includegraphics[width=0.15\linewidth]{plots/population_timings_c2_D=2_NON_DOM=512_c=1.0.png}};
\node[image,right=of frame12] (frame13)
    {\includegraphics[width=0.15\linewidth]{plots/population_timings_c3_D=2_NON_DOM=512_c=1.0.png}};
\node[image,right=of frame13] (frame14)
    {\includegraphics[width=0.15\linewidth]{plots/population_timings_c4_D=2_NON_DOM=512_c=1.0.png}};
\node[image,right=of frame14] (frame15)
    {\includegraphics[width=0.15\linewidth]{plots/population_timings_c5_D=2_NON_DOM=512_c=1.0.png}};
\node[image,right=of frame15] (frame16)
    {\includegraphics[width=0.15\linewidth]{plots/population_timings_c6_D=2_NON_DOM=512_c=1.0.png}};

% Timing 2048, c=1.0
\node [image] (frame21) [image,below = of frame11]
    {\includegraphics[width=0.15\linewidth]{plots/population_timings_c1_D=2_NON_DOM=2048_c=1.0.png}};
    \node[left = of frame21, node distance=0cm and 0cm, rotate=90, anchor=center] {\tiny{\textbf{Timings (ms)}}};
    \node[below = of frame21, node distance=0cm and 0cm] {\tiny{\textbf{Population Size}}};
\node [image,right=of frame21] (frame22) 
    {\includegraphics[width=0.15\linewidth]{plots/population_timings_c2_D=2_NON_DOM=2048_c=1.0.png}};
    \node[below = of frame22, node distance=0cm and 0cm] {\tiny{\textbf{Population Size}}};
\node[image,right=of frame22] (frame23)
    {\includegraphics[width=0.15\linewidth]{plots/population_timings_c3_D=2_NON_DOM=2048_c=1.0.png}};
    \node[below = of frame23, node distance=0cm and 0cm] {\tiny{\textbf{Population Size}}};
\node[image,right=of frame23] (frame24)
    {\includegraphics[width=0.15\linewidth]{plots/population_timings_c4_D=2_NON_DOM=2048_c=1.0.png}};
    \node[below = of frame24, node distance=0cm and 0cm] {\tiny{\textbf{Population Size}}};
\node[image,right=of frame24] (frame25)
    {\includegraphics[width=0.15\linewidth]{plots/population_timings_c5_D=2_NON_DOM=2048_c=1.0.png}};
    \node[below = of frame25, node distance=0cm and 0cm] {\tiny{\textbf{Population Size}}};
\node[image,right=of frame25] (frame26)
    {\includegraphics[width=0.15\linewidth]{plots/population_timings_c6_D=2_NON_DOM=2048_c=1.0.png}};
    \node[below = of frame26, node distance=0cm and 0cm] {\tiny{\textbf{Population Size}}};

\end{tikzpicture}
    \centering
    \includegraphics[height=0.25cm]{legend}
    \caption{\textbf{Mean time taken to update data structure per operation. Results for population,} c = 1.0\textbf{. Top row:} Elite = 128\textbf{. Middle row:} Elite = 512\textbf{. Bottom row:} Elite = 2048\textbf{.}}
    \label{fig:Timing_c=1.0}
\end{figure}

%$C_5$ and $C_6$ perform the best for all simulations
%Simulations 2 and 4 are consistent across all plots
%Simulations 1 and 3 are worst in $C_4$ and slower in $C_{1-3}$, more discrete
%Simulations 1 and 3 are worse than 2 and 4 to start with in $C_5$ and $C_6$ for 128
%the difference between simulation 1,3 and 2,4 decreases as elite set size increases

% timing comparison %
When comparing the different guardian assignment techniques in Figure \ref{fig:comparisons_c=1.0}, we can see that for all four simulations on $C_5$ and $C_6$ perform the best as their rate of growth is lowest for all plots. $S_2$ and $S_4$ have a consistent growth rate across all plots, showing the guardian assignment method has little effect on the overall time taken to update the archive. This does not hold for $S_1$ and $S_3$. These two simulations edit members from the entire population, rather than just the elite set, therefore are slower to converge towards the true Pareto front. The growth curves appear to follow a more discrete nature suggesting large jumps are made in place of continuous growth. This may be down to the emergence of more non dominated solutions. This is in sharp contrast to $C_4$ which sees a higher rate of continuous growth for $S_1$ and $S_3$, and shows less signs of slowing down. This increased time taken compared to $S_2$ and $S_4$ is down to elite members acting as the guardian dominator for orders of magnitude more solutions as guardians are assigned using a greedy approach, therefore the first member of the elite set is likely to have a large number of children. This is supported by the same trend in $C_{1-3}$, except it is amplified for $C_4$ as a greedy approach is used at every option. For a true elite size of $128$, it can be seen that $S_1$ and $S_3$ in populations $C_5$ and $C_6$ initially grow at a quicker rate, but resume the same growth rate as their counterparts by the end of the population. As the elite set increases in size, there is shown to be very little disparity between plots, however there does appear to be a smaller difference between the time taken at each population interval.

%%%%%%%%%%%%%
% Timing
%   timing128_c=0.5
%   timing512_c=0.5
%   timing1024_c=0.5
%%%%%%%%%%%%%
\begin{figure}[h]
    \centering
    \begin{tikzpicture}[image/.style = {inner sep=0pt, outer sep=0pt}, node distance = 1mm and 1mm] 
% Timing 128, c=0.5
\node [image] (frame1)
    {\includegraphics[width=0.15\linewidth]{plots/population_timings_c1_D=2_NON_DOM=128_c=0.5.png}};
    \node[left = of frame1, node distance=0cm and 0cm, rotate=90, anchor=center] {\tiny{\textbf{Timings (ms)}}};
    \node[above = of frame1, node distance=0cm and 0cm] {$C_1$};
\node [image,right=of frame1] (frame2) 
    {\includegraphics[width=0.15\linewidth]{plots/population_timings_c2_D=2_NON_DOM=128_c=0.5.png}};
    \node[above = of frame2, node distance=0cm and 0cm] {$C_2$};
\node[image,right=of frame2] (frame3)
    {\includegraphics[width=0.15\linewidth]{plots/population_timings_c3_D=2_NON_DOM=128_c=0.5.png}};
    \node[above = of frame3, node distance=0cm and 0cm] {$C_3$};
\node[image,right=of frame3] (frame4)
    {\includegraphics[width=0.15\linewidth]{plots/population_timings_c4_D=2_NON_DOM=128_c=0.5.png}};
    \node[above = of frame4, node distance=0cm and 0cm] {$C_4$};
\node[image,right=of frame4] (frame5)
    {\includegraphics[width=0.15\linewidth]{plots/population_timings_c5_D=2_NON_DOM=128_c=0.5.png}};
    \node[above = of frame5, node distance=0cm and 0cm] {$C_5$};
\node[image,right=of frame5] (frame6)
    {\includegraphics[width=0.15\linewidth]{plots/population_timings_c6_D=2_NON_DOM=128_c=0.5.png}};
    \node[above = of frame6, node distance=0cm and 0cm] {$C_6$};

% Timing 512, c=0.5
\node [image] (frame11) [image,below = of frame1]
    {\includegraphics[width=0.15\linewidth]{plots/population_timings_c1_D=2_NON_DOM=512_c=0.5.png}};
    \node[left = of frame11, node distance=0cm and 0cm, rotate=90, anchor=center] {\tiny{\textbf{Timings (ms)}}};
\node [image,right=of frame11] (frame12) 
    {\includegraphics[width=0.15\linewidth]{plots/population_timings_c2_D=2_NON_DOM=512_c=0.5.png}};
\node[image,right=of frame12] (frame13)
    {\includegraphics[width=0.15\linewidth]{plots/population_timings_c3_D=2_NON_DOM=512_c=0.5.png}};
\node[image,right=of frame13] (frame14)
    {\includegraphics[width=0.15\linewidth]{plots/population_timings_c4_D=2_NON_DOM=512_c=0.5.png}};
\node[image,right=of frame14] (frame15)
    {\includegraphics[width=0.15\linewidth]{plots/population_timings_c5_D=2_NON_DOM=512_c=0.5.png}};
\node[image,right=of frame15] (frame16)
    {\includegraphics[width=0.15\linewidth]{plots/population_timings_c6_D=2_NON_DOM=512_c=0.5.png}};

% Timing 1024, c=0.5
\node [image] (frame21) [image,below = of frame11]
    {\includegraphics[width=0.15\linewidth]{plots/population_timings_c1_D=2_NON_DOM=2048_c=0.5.png}};
    \node[left = of frame21, node distance=0cm and 0cm, rotate=90, anchor=center] {\tiny{\textbf{Timings (ms)}}};
    \node[below = of frame21, node distance=0cm and 0cm] {\tiny{\textbf{Population Size}}};
\node [image,right=of frame21] (frame22) 
    {\includegraphics[width=0.15\linewidth]{plots/population_timings_c2_D=2_NON_DOM=2048_c=0.5.png}};
    \node[below = of frame22, node distance=0cm and 0cm] {\tiny{\textbf{Population Size}}};
\node[image,right=of frame22] (frame23)
    {\includegraphics[width=0.15\linewidth]{plots/population_timings_c3_D=2_NON_DOM=2048_c=0.5.png}};
    \node[below = of frame23, node distance=0cm and 0cm] {\tiny{\textbf{Population Size}}};
\node[image,right=of frame23] (frame24)
    {\includegraphics[width=0.15\linewidth]{plots/population_timings_c4_D=2_NON_DOM=2048_c=0.5.png}};
    \node[below = of frame24, node distance=0cm and 0cm] {\tiny{\textbf{Population Size}}};
\node[image,right=of frame24] (frame25)
    {\includegraphics[width=0.15\linewidth]{plots/population_timings_c5_D=2_NON_DOM=2048_c=0.5.png}};
    \node[below = of frame25, node distance=0cm and 0cm] {\tiny{\textbf{Population Size}}};
\node[image,right=of frame25] (frame26)
    {\includegraphics[width=0.15\linewidth]{plots/population_timings_c6_D=2_NON_DOM=2048_c=0.5.png}};
    \node[below = of frame26, node distance=0cm and 0cm] {\tiny{\textbf{Population Size}}};

\end{tikzpicture}
    \centering
    \includegraphics[height=0.25cm]{legend}
    \caption{\textbf{Mean time taken to update data structure per operation. Results for population,} c = 0.5\textbf{. Top row:} Elite = 128\textbf{. Middle row:} Elite = 512\textbf{. Bottom row:} Elite = 2048\textbf{.}}
    \label{fig:Timing_c=0.5}
\end{figure}

When the proportion of non-dominated points is spread later throughout the sequence as in Figure \ref{fig:Timing_c=0.5}, the rate of growth can be seen to be near identical showing that the data structure is resilient to when the true Pareto front appears in the sequence. Same again when the non-dominated solutions appear earlier in the sequence as in Figure \ref{fig:Timing_c=1.5}.\\
It is of note, that in $C_5$ the cost of calculating the Euclidean distance between solutions has not had an effect on timing, therefore the guardian assignment technique should not be discarded in fear of increased computation, when it can be seen to produce higher quality guardian assignments.

%%%%%%%%%%%%%
% Timing
%   timing128_c=1.5
%   timing512_c=1.5
%   timing2048_c=1.5
%%%%%%%%%%%%%
\begin{figure}[H]
    \centering
    \begin{tikzpicture}[image/.style = {inner sep=0pt, outer sep=0pt}, node distance = 1mm and 1mm] 
% Timing 128, c=1.5
\node [image] (frame1)
    {\includegraphics[width=0.15\linewidth]{plots/population_timings_c1_D=2_NON_DOM=128_c=1.5.png}};
    \node[left = of frame1, node distance=0cm and 0cm, rotate=90, anchor=center] {\tiny{\textbf{Timings (ms)}}};
    \node[above = of frame1, node distance=0cm and 0cm] {$C_1$};
\node [image,right=of frame1] (frame2) 
    {\includegraphics[width=0.15\linewidth]{plots/population_timings_c2_D=2_NON_DOM=128_c=1.5.png}};
    \node[above = of frame2, node distance=0cm and 0cm] {$C_2$};
\node[image,right=of frame2] (frame3)
    {\includegraphics[width=0.15\linewidth]{plots/population_timings_c3_D=2_NON_DOM=128_c=1.5.png}};
    \node[above = of frame3, node distance=0cm and 0cm] {$C_3$};
\node[image,right=of frame3] (frame4)
    {\includegraphics[width=0.15\linewidth]{plots/population_timings_c4_D=2_NON_DOM=128_c=1.5.png}};
    \node[above = of frame4, node distance=0cm and 0cm] {$C_4$};
\node[image,right=of frame4] (frame5)
    {\includegraphics[width=0.15\linewidth]{plots/population_timings_c5_D=2_NON_DOM=128_c=1.5.png}};
    \node[above = of frame5, node distance=0cm and 0cm] {$C_5$};
\node[image,right=of frame5] (frame6)
    {\includegraphics[width=0.15\linewidth]{plots/population_timings_c6_D=2_NON_DOM=128_c=1.5.png}};
    \node[above = of frame6, node distance=0cm and 0cm] {$C_6$};

% Timing 512, c=1.5
\node [image] (frame11) [image,below = of frame1]
    {\includegraphics[width=0.15\linewidth]{plots/population_timings_c1_D=2_NON_DOM=512_c=1.5.png}};
    \node[left = of frame11, node distance=0cm and 0cm, rotate=90, anchor=center] {\tiny{\textbf{Timings (ms)}}};
\node [image,right=of frame11] (frame12) 
    {\includegraphics[width=0.15\linewidth]{plots/population_timings_c2_D=2_NON_DOM=512_c=1.5.png}};
\node[image,right=of frame12] (frame13)
    {\includegraphics[width=0.15\linewidth]{plots/population_timings_c3_D=2_NON_DOM=512_c=1.5.png}};
\node[image,right=of frame13] (frame14)
    {\includegraphics[width=0.15\linewidth]{plots/population_timings_c4_D=2_NON_DOM=512_c=1.5.png}};
\node[image,right=of frame14] (frame15)
    {\includegraphics[width=0.15\linewidth]{plots/population_timings_c5_D=2_NON_DOM=512_c=1.5.png}};
\node[image,right=of frame15] (frame16)
    {\includegraphics[width=0.15\linewidth]{plots/population_timings_c6_D=2_NON_DOM=512_c=1.5.png}};

% Timing 1024, c=1.5
\node [image] (frame21) [image,below = of frame11]
    {\includegraphics[width=0.15\linewidth]{plots/population_timings_c1_D=2_NON_DOM=2048_c=1.5.png}};
    \node[left = of frame21, node distance=0cm and 0cm, rotate=90, anchor=center] {\tiny{\textbf{Timings (ms)}}};
    \node[below = of frame21, node distance=0cm and 0cm] {\tiny{\textbf{Population Size}}};
\node [image,right=of frame21] (frame22) 
    {\includegraphics[width=0.15\linewidth]{plots/population_timings_c2_D=2_NON_DOM=2048_c=1.5.png}};
    \node[below = of frame22, node distance=0cm and 0cm] {\tiny{\textbf{Population Size}}};
\node[image,right=of frame22] (frame23)
    {\includegraphics[width=0.15\linewidth]{plots/population_timings_c3_D=2_NON_DOM=2048_c=1.5.png}};
    \node[below = of frame23, node distance=0cm and 0cm] {\tiny{\textbf{Population Size}}};
\node[image,right=of frame23] (frame24)
    {\includegraphics[width=0.15\linewidth]{plots/population_timings_c4_D=2_NON_DOM=2048_c=1.5.png}};
    \node[below = of frame24, node distance=0cm and 0cm] {\tiny{\textbf{Population Size}}};
\node[image,right=of frame24] (frame25)
    {\includegraphics[width=0.15\linewidth]{plots/population_timings_c5_D=2_NON_DOM=2048_c=1.5.png}};
    \node[below = of frame25, node distance=0cm and 0cm] {\tiny{\textbf{Population Size}}};
\node[image,right=of frame25] (frame26)
    {\includegraphics[width=0.15\linewidth]{plots/population_timings_c6_D=2_NON_DOM=2048_c=1.5.png}};
    \node[below = of frame26, node distance=0cm and 0cm] {\tiny{\textbf{Population Size}}};

\end{tikzpicture}
    \centering
    \includegraphics[height=0.25cm]{legend}
    \caption{\textbf{Mean time taken to update data structure per operation. Results for population,} c = 1.5\textbf{. Top row:} Elite = 128\textbf{. Middle row:} Elite = 512\textbf{. Bottom row:} Elite = 2048\textbf{.}}
    \label{fig:Timing_c=1.5}
\end{figure}

\subsubsection{Comparisons}
% number of comparisons, comparison %
For Figures \ref{fig:comparisons_c=0.5}-\ref{fig:comparisons_c=1.5} the logarithmic scale starts at $y=10^6$ as the rate of growth is identical for all simulations and plots when the population size is small. It has been chosen to display the number of comparisons taken as this is the expected primary bottleneck to performance, which was supported when the data structure was analysed under a profiler. Only when the population grows in size, does the difference in comparisons become apparent. The growth follows a more discrete pattern as there are a multitude of operations being performed, some of which will take significantly more comparisons than others.

%%%%%%%%%%%%%
% Comparisons
%   Comparisons128_c=1.0
%   Comparisons512_c=1.0
%   Comparisons2048_c=1.0
%%%%%%%%%%%%%
\begin{figure}[h]
    \centering
    \begin{tikzpicture}[image/.style = {inner sep=0pt, outer sep=0pt}, node distance = 1mm and 1mm] 
% Comparisons 512, c=0.5
\node [image] (frame1)
    {\includegraphics[width=0.15\linewidth]{plots/population_comparisons_c1_D=2_NON_DOM=128_c=1.0.png}};
    \node[left = of frame1, node distance=0cm and 0cm, rotate=90, anchor=center] {\tiny{\textbf{Comparisons}}};
    \node[above = of frame1, node distance=0cm and 0cm] {$C_1$};
\node [image,right=of frame1] (frame2) 
    {\includegraphics[width=0.15\linewidth]{plots/population_comparisons_c2_D=2_NON_DOM=128_c=1.0.png}};
    \node[above = of frame2, node distance=0cm and 0cm] {$C_2$};
\node[image,right=of frame2] (frame3)
    {\includegraphics[width=0.15\linewidth]{plots/population_comparisons_c3_D=2_NON_DOM=128_c=1.0.png}};
    \node[above = of frame3, node distance=0cm and 0cm] {$C_3$};
\node[image,right=of frame3] (frame4)
    {\includegraphics[width=0.15\linewidth]{plots/population_comparisons_c4_D=2_NON_DOM=128_c=1.0.png}};
    \node[above = of frame4, node distance=0cm and 0cm] {$C_4$};
\node[image,right=of frame4] (frame5)
    {\includegraphics[width=0.15\linewidth]{plots/population_comparisons_c5_D=2_NON_DOM=128_c=1.0.png}};
    \node[above = of frame5, node distance=0cm and 0cm] {$C_5$};
\node[image,right=of frame5] (frame6)
    {\includegraphics[width=0.15\linewidth]{plots/population_comparisons_c6_D=2_NON_DOM=128_c=1.0.png}};
    \node[above = of frame6, node distance=0cm and 0cm] {$C_6$};

% Comparisons 512, c=1.0
\node [image] (frame11) [image,below = of frame1]
    {\includegraphics[width=0.15\linewidth]{plots/population_comparisons_c1_D=2_NON_DOM=512_c=1.0.png}};
    \node[left = of frame11, node distance=0cm and 0cm, rotate=90, anchor=center] {\tiny{\textbf{Comparisons}}};
\node [image,right=of frame11] (frame12) 
    {\includegraphics[width=0.15\linewidth]{plots/population_comparisons_c2_D=2_NON_DOM=512_c=1.0.png}};
\node[image,right=of frame12] (frame13)
    {\includegraphics[width=0.15\linewidth]{plots/population_comparisons_c3_D=2_NON_DOM=512_c=1.0.png}};
\node[image,right=of frame13] (frame14)
    {\includegraphics[width=0.15\linewidth]{plots/population_comparisons_c4_D=2_NON_DOM=512_c=1.0.png}};
\node[image,right=of frame14] (frame15)
    {\includegraphics[width=0.15\linewidth]{plots/population_comparisons_c5_D=2_NON_DOM=512_c=1.0.png}};
\node[image,right=of frame15] (frame16)
    {\includegraphics[width=0.15\linewidth]{plots/population_comparisons_c6_D=2_NON_DOM=512_c=1.0.png}};

% Comparisons 2048, c=1.0
\node [image] (frame21) [image,below = of frame11]
    {\includegraphics[width=0.15\linewidth]{plots/population_comparisons_c1_D=2_NON_DOM=2048_c=1.0.png}};
    \node[left = of frame21, node distance=0cm and 0cm, rotate=90, anchor=center] {\tiny{\textbf{Comparisons}}};
    \node[below = of frame21, node distance=0cm and 0cm] {\tiny{\textbf{Population Size}}};
\node [image,right=of frame21] (frame22) 
    {\includegraphics[width=0.15\linewidth]{plots/population_comparisons_c2_D=2_NON_DOM=2048_c=1.0.png}};
    \node[below = of frame22, node distance=0cm and 0cm] {\tiny{\textbf{Population Size}}};
\node[image,right=of frame22] (frame23)
    {\includegraphics[width=0.15\linewidth]{plots/population_comparisons_c3_D=2_NON_DOM=2048_c=1.0.png}};
    \node[below = of frame23, node distance=0cm and 0cm] {\tiny{\textbf{Population Size}}};
\node[image,right=of frame23] (frame24)
    {\includegraphics[width=0.15\linewidth]{plots/population_comparisons_c4_D=2_NON_DOM=2048_c=1.0.png}};
    \node[below = of frame24, node distance=0cm and 0cm] {\tiny{\textbf{Population Size}}};
\node[image,right=of frame24] (frame25)
    {\includegraphics[width=0.15\linewidth]{plots/population_comparisons_c5_D=2_NON_DOM=2048_c=1.0.png}};
    \node[below = of frame25, node distance=0cm and 0cm] {\tiny{\textbf{Population Size}}};
\node[image,right=of frame25] (frame26)
    {\includegraphics[width=0.15\linewidth]{plots/population_comparisons_c6_D=2_NON_DOM=2048_c=1.0.png}};
    \node[below = of frame26, node distance=0cm and 0cm] {\tiny{\textbf{Population Size}}};

\end{tikzpicture}
    \centering
    \includegraphics[height=0.25cm]{legend}
    \caption{\textbf{The cumulative number of comparisons required to update the data structure per operation. Results for population,} c = 1.0\textbf{. Top row:} Elite = 128\textbf{. Middle row:} Elite = 512\textbf{. Bottom row:} Elite = 2048\textbf{.}}
    \label{fig:comparisons_c=1.0}
\end{figure}

%Simulations 2 and 4 are identical for $C_{1-4}, c4 is slightly higher$ 
%Simulations 2 and 4 finish marginally higher in $C_4$
%Simulation 4 performs better than 2 in in $C_5$ but only for elite =512 or =1024, is the same for =128

%Simulations 1 and 3 have discrete growth in $C_{1-3}$
%The rate of growth is much higher in $C_4$, bad performance
%Simulations 1 and 3 actually perform better in $C_5$ and $C_6$ than counterparts, rate of growth is much lower, they are better suited, difference is more apparent in the smaller elite archive sizes
Figures \ref{fig:comparisons_c=0.5}-\ref{fig:comparisons_c=1.5} show the cumulative number of domination comparisons required at each time step. Upon comparing the different guardian assignment techniques, we can see a clear distinction between the number of comparisons required for the different simulations. It is immediately clear that the number of domination comparisons does not follow the time taken to perform an update equally. $S_2$ and $S_4$ have identical growth rates for $C_{1-3,6}$ whereas in $C_4$ the growth rate is marginally higher. $S_2$ is consistent with the other plots for $C_5$, unlike $S_4$ which actually performs better when there are 512 solutions in the elite set. This shows there is a consistency to performance when members of the elite set are regularly changed.\\
$S_1$ and $S_3$ follow a discrete growth pattern for $C_{1-3}$. This is likely down to their guardian assignment taking a greedy approach where possible, and choosing the guardian as an immediate predecessor if possible, which has a best case complexity of $O(1)$. This leads to a much more linear growth trend which is also greater than in $S_2$ and $S_4$. This continues for $C_4$, which takes significantly more dominance comparisons per time step than its counterparts. This trend is reversed however in $C_5$ and $C_6$ where $S_1$ and $S_3$ give a noticeably lower growth rate. This difference is more apparent when the archive size is smaller. This increased performance is likely caused by a less concentrated number of children for members of the elite set, thus reducing the cost of an edit operation, when every child needs reassigning a guardian. These two simulations also edit members from the entire population, rather than just the elite set, and are therefore more likely to converge faster towards the true front.

%%%%%%%%%%%%%
% Comparisons
%   Comparisons128_c=0.5
%   Comparisons512_c=0.5
%   Comparisons2048_c=0.5
%%%%%%%%%%%%%
\begin{figure}[h]
    \centering
    \begin{tikzpicture}[image/.style = {inner sep=0pt, outer sep=0pt}, node distance = 1mm and 1mm] 
% Comparisons 128, c=0.5
\node [image] (frame1)
    {\includegraphics[width=0.15\linewidth]{plots/population_comparisons_c1_D=2_NON_DOM=128_c=0.5.png}};
    \node[left = of frame1, node distance=0cm and 0cm, rotate=90, anchor=center] {\tiny{\textbf{Comparisons}}};
    \node[above = of frame1, node distance=0cm and 0cm] {$C_1$};
\node [image,right=of frame1] (frame2) 
    {\includegraphics[width=0.15\linewidth]{plots/population_comparisons_c2_D=2_NON_DOM=128_c=0.5.png}};
    \node[above = of frame2, node distance=0cm and 0cm] {$C_2$};
\node[image,right=of frame2] (frame3)
    {\includegraphics[width=0.15\linewidth]{plots/population_comparisons_c3_D=2_NON_DOM=128_c=0.5.png}};
    \node[above = of frame3, node distance=0cm and 0cm] {$C_3$};
\node[image,right=of frame3] (frame4)
    {\includegraphics[width=0.15\linewidth]{plots/population_comparisons_c4_D=2_NON_DOM=128_c=0.5.png}};
    \node[above = of frame4, node distance=0cm and 0cm] {$C_4$};
\node[image,right=of frame4] (frame5)
    {\includegraphics[width=0.15\linewidth]{plots/population_comparisons_c5_D=2_NON_DOM=128_c=0.5.png}};
    \node[above = of frame5, node distance=0cm and 0cm] {$C_5$};
\node[image,right=of frame5] (frame6)
    {\includegraphics[width=0.15\linewidth]{plots/population_comparisons_c6_D=2_NON_DOM=128_c=0.5.png}};
    \node[above = of frame6, node distance=0cm and 0cm] {$C_6$};

% Comparisons 512, c=0.5
\node [image] (frame11) [image,below = of frame1]
    {\includegraphics[width=0.15\linewidth]{plots/population_comparisons_c1_D=2_NON_DOM=512_c=0.5.png}};
    \node[left = of frame11, node distance=0cm and 0cm, rotate=90, anchor=center] {\tiny{\textbf{Comparisons}}};
\node [image,right=of frame11] (frame12) 
    {\includegraphics[width=0.15\linewidth]{plots/population_comparisons_c2_D=2_NON_DOM=512_c=0.5.png}};
\node[image,right=of frame12] (frame13)
    {\includegraphics[width=0.15\linewidth]{plots/population_comparisons_c3_D=2_NON_DOM=512_c=0.5.png}};
\node[image,right=of frame13] (frame14)
    {\includegraphics[width=0.15\linewidth]{plots/population_comparisons_c4_D=2_NON_DOM=512_c=0.5.png}};
\node[image,right=of frame14] (frame15)
    {\includegraphics[width=0.15\linewidth]{plots/population_comparisons_c5_D=2_NON_DOM=512_c=0.5.png}};
\node[image,right=of frame15] (frame16)
    {\includegraphics[width=0.15\linewidth]{plots/population_comparisons_c6_D=2_NON_DOM=512_c=0.5.png}};

% Comparisons 1024, c=0.5
\node [image] (frame21) [image,below = of frame11]
    {\includegraphics[width=0.15\linewidth]{plots/population_comparisons_c1_D=2_NON_DOM=2048_c=0.5.png}};
    \node[left = of frame21, node distance=0cm and 0cm, rotate=90, anchor=center] {\tiny{\textbf{Comparisons}}};
    \node[below = of frame21, node distance=0cm and 0cm] {\tiny{\textbf{Population Size}}};
\node [image,right=of frame21] (frame22) 
    {\includegraphics[width=0.15\linewidth]{plots/population_comparisons_c2_D=2_NON_DOM=2048_c=0.5.png}};
    \node[below = of frame22, node distance=0cm and 0cm] {\tiny{\textbf{Population Size}}};
\node[image,right=of frame22] (frame23)
    {\includegraphics[width=0.15\linewidth]{plots/population_comparisons_c3_D=2_NON_DOM=2048_c=0.5.png}};
    \node[below = of frame23, node distance=0cm and 0cm] {\tiny{\textbf{Population Size}}};
\node[image,right=of frame23] (frame24)
    {\includegraphics[width=0.15\linewidth]{plots/population_comparisons_c4_D=2_NON_DOM=2048_c=0.5.png}};
    \node[below = of frame24, node distance=0cm and 0cm] {\tiny{\textbf{Population Size}}};
\node[image,right=of frame24] (frame25)
    {\includegraphics[width=0.15\linewidth]{plots/population_comparisons_c5_D=2_NON_DOM=2048_c=0.5.png}};
    \node[below = of frame25, node distance=0cm and 0cm] {\tiny{\textbf{Population Size}}};
\node[image,right=of frame25] (frame26)
    {\includegraphics[width=0.15\linewidth]{plots/population_comparisons_c6_D=2_NON_DOM=2048_c=0.5.png}};
    \node[below = of frame26, node distance=0cm and 0cm] {\tiny{\textbf{Population Size}}};

\end{tikzpicture}
    \centering
    \includegraphics[height=0.25cm]{legend}
    \caption{\textbf{The cumulative number of comparisons required to update the data structure per operation. Results for population,} c = 0.5\textbf{. Top row:} Elite = 128\textbf{. Middle row:} Elite = 512\textbf{. Bottom row:} Elite = 2048\textbf{.}}
    \label{fig:comparisons_c=0.5}
\end{figure}

When the proportion of non-dominated points is more heavily concentrated earlier in the sequence (Figure \ref{fig:comparisons_c=1.5}) compared to later (Figure \ref{fig:comparisons_c=0.5}), $S_1$ and $S_3$  for $C_{1-3}$, follow the trend of $S_2$ and $S_4$ more closely. This is because they only edit elite solutions, and once the elite set has appeared, this difference in simulation has been narrowed. This would explain why this difference is more apparent for the larger elite sets. This difference did not appear in the timings hence we can conclude that the distribution of points does not affect the overall performance of the data structure.

%%%%%%%%%%%%%
% Comparisons
%   Comparisons128_c=1.5
%   Comparisons512_c=1.5
%   Comparisons2048_c=1.5
%%%%%%%%%%%%%
\begin{figure}[h]
    \centering
    \begin{tikzpicture}[image/.style = {inner sep=0pt, outer sep=0pt}, node distance = 1mm and 1mm] 
% Comparisons 128, c=1.5
\node [image] (frame1)
    {\includegraphics[width=0.15\linewidth]{plots/population_comparisons_c1_D=2_NON_DOM=128_c=1.5.png}};
    \node[left = of frame1, node distance=0cm and 0cm, rotate=90, anchor=center] {\tiny{\textbf{Comparisons}}};
    \node[above = of frame1, node distance=0cm and 0cm] {$C_1$};
\node [image,right=of frame1] (frame2) 
    {\includegraphics[width=0.15\linewidth]{plots/population_comparisons_c2_D=2_NON_DOM=128_c=1.5.png}};
    \node[above = of frame2, node distance=0cm and 0cm] {$C_2$};
\node[image,right=of frame2] (frame3)
    {\includegraphics[width=0.15\linewidth]{plots/population_comparisons_c3_D=2_NON_DOM=128_c=1.5.png}};
    \node[above = of frame3, node distance=0cm and 0cm] {$C_3$};
\node[image,right=of frame3] (frame4)
    {\includegraphics[width=0.15\linewidth]{plots/population_comparisons_c4_D=2_NON_DOM=128_c=1.5.png}};
    \node[above = of frame4, node distance=0cm and 0cm] {$C_4$};
\node[image,right=of frame4] (frame5)
    {\includegraphics[width=0.15\linewidth]{plots/population_comparisons_c5_D=2_NON_DOM=128_c=1.5.png}};
    \node[above = of frame5, node distance=0cm and 0cm] {$C_5$};
\node[image,right=of frame5] (frame6)
    {\includegraphics[width=0.15\linewidth]{plots/population_comparisons_c6_D=2_NON_DOM=128_c=1.5.png}};
    \node[above = of frame6, node distance=0cm and 0cm] {$C_6$};

% Comparisons 512, c=1.5
\node [image] (frame11) [image,below = of frame1]
    {\includegraphics[width=0.15\linewidth]{plots/population_comparisons_c1_D=2_NON_DOM=512_c=1.5.png}};
    \node[left = of frame11, node distance=0cm and 0cm, rotate=90, anchor=center] {\tiny{\textbf{Comparisons}}};
\node [image,right=of frame11] (frame12) 
    {\includegraphics[width=0.15\linewidth]{plots/population_comparisons_c2_D=2_NON_DOM=512_c=1.5.png}};
\node[image,right=of frame12] (frame13)
    {\includegraphics[width=0.15\linewidth]{plots/population_comparisons_c3_D=2_NON_DOM=512_c=1.5.png}};
\node[image,right=of frame13] (frame14)
    {\includegraphics[width=0.15\linewidth]{plots/population_comparisons_c4_D=2_NON_DOM=512_c=1.5.png}};
\node[image,right=of frame14] (frame15)
    {\includegraphics[width=0.15\linewidth]{plots/population_comparisons_c5_D=2_NON_DOM=512_c=1.5.png}};
\node[image,right=of frame15] (frame16)
    {\includegraphics[width=0.15\linewidth]{plots/population_comparisons_c6_D=2_NON_DOM=512_c=1.5.png}};

% Comparisons 2048, c=1.5
\node [image] (frame21) [image,below = of frame11]
    {\includegraphics[width=0.15\linewidth]{plots/population_comparisons_c1_D=2_NON_DOM=2048_c=1.5.png}};
    \node[left = of frame21, node distance=0cm and 0cm, rotate=90, anchor=center] {\tiny{\textbf{Comparisons}}};
    \node[below = of frame21, node distance=0cm and 0cm] {\tiny{\textbf{Population Size}}};
\node [image,right=of frame21] (frame22) 
    {\includegraphics[width=0.15\linewidth]{plots/population_comparisons_c2_D=2_NON_DOM=2048_c=1.5.png}};
    \node[below = of frame22, node distance=0cm and 0cm] {\tiny{\textbf{Population Size}}};
\node[image,right=of frame22] (frame23)
    {\includegraphics[width=0.15\linewidth]{plots/population_comparisons_c3_D=2_NON_DOM=2048_c=1.5.png}};
    \node[below = of frame23, node distance=0cm and 0cm] {\tiny{\textbf{Population Size}}};
\node[image,right=of frame23] (frame24)
    {\includegraphics[width=0.15\linewidth]{plots/population_comparisons_c4_D=2_NON_DOM=2048_c=1.5.png}};
    \node[below = of frame24, node distance=0cm and 0cm] {\tiny{\textbf{Population Size}}};
\node[image,right=of frame24] (frame25)
    {\includegraphics[width=0.15\linewidth]{plots/population_comparisons_c5_D=2_NON_DOM=2048_c=1.5.png}};
    \node[below = of frame25, node distance=0cm and 0cm] {\tiny{\textbf{Population Size}}};
\node[image,right=of frame25] (frame26)
    {\includegraphics[width=0.15\linewidth]{plots/population_comparisons_c6_D=2_NON_DOM=2048_c=1.5.png}};
    \node[below = of frame26, node distance=0cm and 0cm] {\tiny{\textbf{Population Size}}};

\end{tikzpicture}
    \centering
    \includegraphics[height=0.25cm]{legend}
    \caption{\textbf{The cumulative number of comparisons required to update the data structure per operation. Results for population,} c = 1.5\textbf{. Top row:} Elite = 128 \textbf{. Middle row:} Elite = 512 \textbf{. Bottom row:} Elite = 2048\textbf{.}}
    \label{fig:comparisons_c=1.5}
\end{figure}

\subsubsection{Elite Set}
% explain elite
Figures \ref{fig:elite_c=1.0}-\ref{fig:elite_c=1.5} show the cumulative number of comparisons taken, when the elite set's size has changed after an edit operation. This was chosen to be graphed because the change in elite set is fundamental to the solving of a MOOP, thus understanding the characteristics of such an elite set is beneficial to being able to accurately utilise the data structure. If updating the elite set is significantly expensive, it would be inappropriate to use the archive on a problem which has a rapidly moving elite set. Markers have been placed every 50 data points. There is a difference in the number of markers per simulation due to whether the elite set is constructed primarily through add or edit operations as only data for edit operations are recorded here.

%%%%%%%%%%%%%
% Elite
%   Elite128_c=1.0
%   Elite512_c=1.0
%   Elite2048_c=1.0
%%%%%%%%%%%%%
\begin{figure}[H]
    \centering
    \begin{tikzpicture}[image/.style = {inner sep=0pt, outer sep=0pt}, node distance = 1mm and 1mm] 
% Elite 128, c=1.0
\node [image] (frame1)
    {\includegraphics[width=0.15\linewidth]{plots/population_eliteComparisons_c1_D=2_NON_DOM=128_c=1.0.png}};
    \node[left = of frame1, node distance=0cm and 0cm, rotate=90, anchor=center] {\tiny{\textbf{Comparisons}}};
    \node[above = of frame1, node distance=0cm and 0cm] {$C_1$};
\node [image,right=of frame1] (frame2) 
    {\includegraphics[width=0.15\linewidth]{plots/population_eliteComparisons_c2_D=2_NON_DOM=128_c=1.0.png}};
    \node[above = of frame2, node distance=0cm and 0cm] {$C_2$};
\node[image,right=of frame2] (frame3)
    {\includegraphics[width=0.15\linewidth]{plots/population_eliteComparisons_c3_D=2_NON_DOM=128_c=1.0.png}};
    \node[above = of frame3, node distance=0cm and 0cm] {$C_3$};
\node[image,right=of frame3] (frame4)
    {\includegraphics[width=0.15\linewidth]{plots/population_eliteComparisons_c4_D=2_NON_DOM=128_c=1.0.png}};
    \node[above = of frame4, node distance=0cm and 0cm] {$C_4$};
\node[image,right=of frame4] (frame5)
    {\includegraphics[width=0.15\linewidth]{plots/population_eliteComparisons_c5_D=2_NON_DOM=128_c=1.0.png}};
    \node[above = of frame5, node distance=0cm and 0cm] {$C_5$};
\node[image,right=of frame5] (frame6)
    {\includegraphics[width=0.15\linewidth]{plots/population_eliteComparisons_c6_D=2_NON_DOM=128_c=1.0.png}};
    \node[above = of frame6, node distance=0cm and 0cm] {$C_6$};

% Elite 512, c=1.0
\node [image] (frame11) [image,below = of frame1]
    {\includegraphics[width=0.15\linewidth]{plots/population_eliteComparisons_c1_D=2_NON_DOM=512_c=1.0.png}};
    \node[left = of frame11, node distance=0cm and 0cm, rotate=90, anchor=center] {\tiny{\textbf{Comparisons}}};
\node [image,right=of frame11] (frame12) 
    {\includegraphics[width=0.15\linewidth]{plots/population_eliteComparisons_c2_D=2_NON_DOM=512_c=1.0.png}};
\node[image,right=of frame12] (frame13)
    {\includegraphics[width=0.15\linewidth]{plots/population_eliteComparisons_c3_D=2_NON_DOM=512_c=1.0.png}};
\node[image,right=of frame13] (frame14)
    {\includegraphics[width=0.15\linewidth]{plots/population_eliteComparisons_c4_D=2_NON_DOM=512_c=1.0.png}};
\node[image,right=of frame14] (frame15)
    {\includegraphics[width=0.15\linewidth]{plots/population_eliteComparisons_c5_D=2_NON_DOM=512_c=1.0.png}};
\node[image,right=of frame15] (frame16)
    {\includegraphics[width=0.15\linewidth]{plots/population_eliteComparisons_c6_D=2_NON_DOM=512_c=1.0.png}};

% Elite 2048, c=1.0
\node [image] (frame21) [image,below = of frame11]
    {\includegraphics[width=0.15\linewidth]{plots/population_eliteComparisons_c1_D=2_NON_DOM=2048_c=1.0.png}};
    \node[left = of frame21, node distance=0cm and 0cm, rotate=90, anchor=center] {\tiny{\textbf{Comparisons}}};
    \node[below = of frame21, node distance=0cm and 0cm] {\tiny{\textbf{Population Size}}};
\node [image,right=of frame21] (frame22) 
    {\includegraphics[width=0.15\linewidth]{plots/population_eliteComparisons_c2_D=2_NON_DOM=2048_c=1.0.png}};
    \node[below = of frame22, node distance=0cm and 0cm] {\tiny{\textbf{Population Size}}};
\node[image,right=of frame22] (frame23)
    {\includegraphics[width=0.15\linewidth]{plots/population_eliteComparisons_c3_D=2_NON_DOM=2048_c=1.0.png}};
    \node[below = of frame23, node distance=0cm and 0cm] {\tiny{\textbf{Population Size}}};
\node[image,right=of frame23] (frame24)
    {\includegraphics[width=0.15\linewidth]{plots/population_eliteComparisons_c4_D=2_NON_DOM=2048_c=1.0.png}};
    \node[below = of frame24, node distance=0cm and 0cm] {\tiny{\textbf{Population Size}}};
\node[image,right=of frame24] (frame25)
    {\includegraphics[width=0.15\linewidth]{plots/population_eliteComparisons_c5_D=2_NON_DOM=2048_c=1.0.png}};
    \node[below = of frame25, node distance=0cm and 0cm] {\tiny{\textbf{Population Size}}};
\node[image,right=of frame25] (frame26)
    {\includegraphics[width=0.15\linewidth]{plots/population_eliteComparisons_c6_D=2_NON_DOM=2048_c=1.0.png}};
    \node[below = of frame26, node distance=0cm and 0cm] {\tiny{\textbf{Population Size}}};

\end{tikzpicture}
    \centering
    \includegraphics[height=0.25cm]{legend}
    \caption{\textbf{The ratio between the number of comparisons taken to update the elite set per edit operation. Results for population,} c = 1.0\textbf{. Top row:} Elite = 128 \textbf{. Middle row:} Elite = 512 \textbf{. Bottom row:} Elite = 2048\textbf{.}}
    \label{fig:elite_c=1.0}
\end{figure}

% elite analysis
 % Simulation 1
% all start very low then quickly increase unlike other simulations
% uses the least comparisons in every population, gets close to others for 2048, and c4
% has alot of variation between combinations
% 128 produces the best
% c2 best combination
% c4 does produce the worse growth rate
% c1 produces the least growth, sudden spikes

% Simulation 2
% second best growth rate
% difference to others is more noticeable in smaller population size
% consistent between c5 and c6
% growth shows seems to dip and then increase again in smaller elite set
% performs best c1 2048

% Simulation 3 and 4
% follow identical trend across all
% simulation 3 does seem to perform slightly better for 2048
% the rate of growth is actually decreasing in every plot
%  c6 shows the best performance

In figure \ref{fig:elite_c=1.0}, for $S_1$ where random solutions are changed to their true objective location, the number of comparisons starts grows at a significantly lower rate than all simulations across all plots. It is also the best performing simulation, however it does have a tendency to reach the same rate of growth as the other simulations when the elite set is 2048. This is particularly noticeable for $C_4$. Overall the smaller 128 elite set performs best for this simulation as the change in growth tends to increases at a flat rate here. The guardian assignment $C_1$ produces the the least growth, however it is still outperformed by $C_2$ for an elite set of 128.\\
$S_2$ has the second best growth rate across all plots, which following the common trait from $S_1$ of moving directly to the true objective vector, shows that the archive performs better when solutions converge faster into the elite set. As the size of the elite set decreases, $S_2$ differs less from $S_3$ and $S_4$. This may be down to larger elite sets not being as fast to converge. The trend for $C_5$ and $C_6$ is very similar. In the smaller elite of the elite sets, the rate of growth correlates stronger with the population size increase. $C_5$ and $C_6$ show the best performance overall across population sizes.\\
$S_3$ and $S_4$ follow a similar trend to each other across all plots. The rate of growth is always at a near constant rate. $S_3$ does have a tendency to minorly outperform $S_4$ which is consistent with $S_1$ outperforming $S_2$, however the difference is less refined. The size of the elite set does not have a noticeable impact on the performance of these simulations. The best performing guardian assignment method is $C_6$.
 
%%%%%%%%%%%%%
% Elite
%   Elite128_c=0.5
%   Elite128_c=0.5
%   Elite128_c=0.5
%%%%%%%%%%%%%
\begin{figure}[H]
    \centering
    \begin{tikzpicture}[image/.style = {inner sep=0pt, outer sep=0pt}, node distance = 1mm and 1mm] 
% Elite 128, c=0.5
\node [image] (frame1)
    {\includegraphics[width=0.15\linewidth]{plots/population_eliteComparisons_c1_D=2_NON_DOM=128_c=0.5.png}};
    \node[left = of frame1, node distance=0cm and 0cm, rotate=90, anchor=center] {\tiny{\textbf{Comparisons}}};
    \node[above = of frame1, node distance=0cm and 0cm] {$C_1$};
\node [image,right=of frame1] (frame2) 
    {\includegraphics[width=0.15\linewidth]{plots/population_eliteComparisons_c2_D=2_NON_DOM=128_c=0.5.png}};
    \node[above = of frame2, node distance=0cm and 0cm] {$C_2$};
\node[image,right=of frame2] (frame3)
    {\includegraphics[width=0.15\linewidth]{plots/population_eliteComparisons_c3_D=2_NON_DOM=128_c=0.5.png}};
    \node[above = of frame3, node distance=0cm and 0cm] {$C_3$};
\node[image,right=of frame3] (frame4)
    {\includegraphics[width=0.15\linewidth]{plots/population_eliteComparisons_c4_D=2_NON_DOM=128_c=0.5.png}};
    \node[above = of frame4, node distance=0cm and 0cm] {$C_4$};
\node[image,right=of frame4] (frame5)
    {\includegraphics[width=0.15\linewidth]{plots/population_eliteComparisons_c5_D=2_NON_DOM=128_c=0.5.png}};
    \node[above = of frame5, node distance=0cm and 0cm] {$C_5$};
\node[image,right=of frame5] (frame6)
    {\includegraphics[width=0.15\linewidth]{plots/population_eliteComparisons_c6_D=2_NON_DOM=128_c=0.5.png}};
    \node[above = of frame6, node distance=0cm and 0cm] {$C_6$};

% Elite 512, c=0.5
\node [image] (frame11) [image,below = of frame1]
    {\includegraphics[width=0.15\linewidth]{plots/population_eliteComparisons_c1_D=2_NON_DOM=512_c=0.5.png}};
    \node[left = of frame11, node distance=0cm and 0cm, rotate=90, anchor=center] {\tiny{\textbf{Comparisons}}};
\node [image,right=of frame11] (frame12) 
    {\includegraphics[width=0.15\linewidth]{plots/population_eliteComparisons_c2_D=2_NON_DOM=512_c=0.5.png}};
\node[image,right=of frame12] (frame13)
    {\includegraphics[width=0.15\linewidth]{plots/population_eliteComparisons_c3_D=2_NON_DOM=512_c=0.5.png}};
\node[image,right=of frame13] (frame14)
    {\includegraphics[width=0.15\linewidth]{plots/population_eliteComparisons_c4_D=2_NON_DOM=512_c=0.5.png}};
\node[image,right=of frame14] (frame15)
    {\includegraphics[width=0.15\linewidth]{plots/population_eliteComparisons_c5_D=2_NON_DOM=512_c=0.5.png}};
\node[image,right=of frame15] (frame16)
    {\includegraphics[width=0.15\linewidth]{plots/population_eliteComparisons_c6_D=2_NON_DOM=512_c=0.5.png}};

% Elite 2048, c=0.5
\node [image] (frame21) [image,below = of frame11]
    {\includegraphics[width=0.15\linewidth]{plots/population_eliteComparisons_c1_D=2_NON_DOM=2048_c=0.5.png}};
    \node[left = of frame21, node distance=0cm and 0cm, rotate=90, anchor=center] {\tiny{\textbf{Comparisons}}};
    \node[below = of frame21, node distance=0cm and 0cm] {\tiny{\textbf{Population Size}}};
\node [image,right=of frame21] (frame22) 
    {\includegraphics[width=0.15\linewidth]{plots/population_eliteComparisons_c2_D=2_NON_DOM=2048_c=0.5.png}};
    \node[below = of frame22, node distance=0cm and 0cm] {\tiny{\textbf{Population Size}}};
\node[image,right=of frame22] (frame23)
    {\includegraphics[width=0.15\linewidth]{plots/population_eliteComparisons_c3_D=2_NON_DOM=2048_c=0.5.png}};
    \node[below = of frame23, node distance=0cm and 0cm] {\tiny{\textbf{Population Size}}};
\node[image,right=of frame23] (frame24)
    {\includegraphics[width=0.15\linewidth]{plots/population_eliteComparisons_c4_D=2_NON_DOM=2048_c=0.5.png}};
    \node[below = of frame24, node distance=0cm and 0cm] {\tiny{\textbf{Population Size}}};
\node[image,right=of frame24] (frame25)
    {\includegraphics[width=0.15\linewidth]{plots/population_eliteComparisons_c5_D=2_NON_DOM=2048_c=0.5.png}};
    \node[below = of frame25, node distance=0cm and 0cm] {\tiny{\textbf{Population Size}}};
\node[image,right=of frame25] (frame26)
    {\includegraphics[width=0.15\linewidth]{plots/population_eliteComparisons_c6_D=2_NON_DOM=2048_c=0.5.png}};
    \node[below = of frame26, node distance=0cm and 0cm] {\tiny{\textbf{Population Size}}};

\end{tikzpicture}
    \centering
    \includegraphics[height=0.25cm]{legend}
    \caption{\textbf{The ratio between the number of comparisons taken to update the elite set per edit operation. Results for population,} c = 0.5\textbf{. Top row:} Elite = 128 \textbf{. Middle row:} Elite = 512 \textbf{. Bottom row:} Elite = 2048\textbf{.}}
    \label{fig:elite_c=0.5}
\end{figure}

% elite c comparison
%SIMULATION 1
%   has slower growth for 0.5, noticeable on the larger sets
%   number of markers is consistent tho showing this is not because of increased edits
%SIMULATION 2
%   follows exact same trend?
%   concentration of markers is reversed but same growth, in larger sets
%SIMULATION 3
%   follows same trend?
%SIMULATION 4
%   follows same trend, concentration of markers is reversed but same growth, more apparent in larger sets

To investigate how the value of $c$ affects the number of comparisons used to update the elite set, Figures \ref{fig:comparisons_c=0.5} and \ref{fig:comparisons_c=1.5} should be compared. When $c=0.5$, non-dominated points have a higher probability of being generated earlier in the sequence, whereas when $c=1.5$, non-dominated points have a tendency to be generated later in the sequence. This does not mean other points cannot become members of the elite set and then later be displaced out.\\
$S_1$ is the most noticeably different simulation. When $c=0.5$ there is a slower rate of growth to begin with which sharply accelerates as the population begins to reach its maximum. This differs from when $c=1.5$ which instead has a sharp immediate growth that starts to flatten to a constant rate once the predicted size of the elite archive is presumably met. The frequency of markers shows that the majority of markers occur earlier when $c=1.5$ and later when $c=0.5$. This correlates to distribution of points within the sequence. The total number of markers does appear to be consistent however, showing that this change in distribution is not caused by a change in the number of edit operations taking place.\\
$S_2$, $S_3$, and $S_4$ all appear to follow the same trend, where there is a reversed concentration of markers, but little to no change in rates of growth. The reversed concentration of markers also only tends to be prevalent in the larger elite sets as for 128 there does not seem to be a shift with the majority of edit operations taking place towards the end of the population's lifespan.\\
It is interesting that only one of the four simulations rate of growth was changed by the distribution of non-dominated points. $S_1$ randomly edits any member of the population, changing the objective vector to the true underlying location. This makes the true Pareto front likely to appear quickly, therefore when the true non-dominated locations appear later in the sequence, the existing front will move forward to the true front, and then the number of comparisons required to perform an update reduces.

%%%%%%%%%%%%%
% Elite
%   Elite128_c=1.5
%   Elite512_c=1.5
%   Elite2048_c=1.5
%%%%%%%%%%%%%
\begin{figure}[H]
    \centering
    \begin{tikzpicture}[image/.style = {inner sep=0pt, outer sep=0pt}, node distance = 1mm and 1mm] 
% Elite 128, c=1.5
\node [image] (frame1)
    {\includegraphics[width=0.15\linewidth]{plots/population_eliteComparisons_c1_D=2_NON_DOM=128_c=1.5.png}};
    \node[left = of frame1, node distance=0cm and 0cm, rotate=90, anchor=center] {\tiny{\textbf{Comparisons}}};
    \node[above = of frame1, node distance=0cm and 0cm] {$C_1$};
\node [image,right=of frame1] (frame2) 
    {\includegraphics[width=0.15\linewidth]{plots/population_eliteComparisons_c2_D=2_NON_DOM=128_c=1.5.png}};
    \node[above = of frame2, node distance=0cm and 0cm] {$C_2$};
\node[image,right=of frame2] (frame3)
    {\includegraphics[width=0.15\linewidth]{plots/population_eliteComparisons_c3_D=2_NON_DOM=128_c=1.5.png}};
    \node[above = of frame3, node distance=0cm and 0cm] {$C_3$};
\node[image,right=of frame3] (frame4)
    {\includegraphics[width=0.15\linewidth]{plots/population_eliteComparisons_c4_D=2_NON_DOM=128_c=1.5.png}};
    \node[above = of frame4, node distance=0cm and 0cm] {$C_4$};
\node[image,right=of frame4] (frame5)
    {\includegraphics[width=0.15\linewidth]{plots/population_eliteComparisons_c5_D=2_NON_DOM=128_c=1.5.png}};
    \node[above = of frame5, node distance=0cm and 0cm] {$C_5$};
\node[image,right=of frame5] (frame6)
    {\includegraphics[width=0.15\linewidth]{plots/population_eliteComparisons_c6_D=2_NON_DOM=128_c=1.5.png}};
    \node[above = of frame6, node distance=0cm and 0cm] {$C_6$};

% Elite 512, c=1.5
\node [image] (frame11) [image,below = of frame1]
    {\includegraphics[width=0.15\linewidth]{plots/population_eliteComparisons_c1_D=2_NON_DOM=512_c=1.5.png}};
    \node[left = of frame11, node distance=0cm and 0cm, rotate=90, anchor=center] {\tiny{\textbf{Comparisons}}};
\node [image,right=of frame11] (frame12) 
    {\includegraphics[width=0.15\linewidth]{plots/population_eliteComparisons_c2_D=2_NON_DOM=512_c=1.5.png}};
\node[image,right=of frame12] (frame13)
    {\includegraphics[width=0.15\linewidth]{plots/population_eliteComparisons_c3_D=2_NON_DOM=512_c=1.5.png}};
\node[image,right=of frame13] (frame14)
    {\includegraphics[width=0.15\linewidth]{plots/population_eliteComparisons_c4_D=2_NON_DOM=512_c=1.5.png}};
\node[image,right=of frame14] (frame15)
    {\includegraphics[width=0.15\linewidth]{plots/population_eliteComparisons_c5_D=2_NON_DOM=512_c=1.5.png}};
\node[image,right=of frame15] (frame16)
    {\includegraphics[width=0.15\linewidth]{plots/population_eliteComparisons_c6_D=2_NON_DOM=512_c=1.5.png}};

% Elite 2048, c=1.5
\node [image] (frame21) [image,below = of frame11]
    {\includegraphics[width=0.15\linewidth]{plots/population_eliteComparisons_c1_D=2_NON_DOM=2048_c=1.5.png}};
    \node[left = of frame21, node distance=0cm and 0cm, rotate=90, anchor=center] {\tiny{\textbf{Comparisons}}};
    \node[below = of frame21, node distance=0cm and 0cm] {\tiny{\textbf{Population Size}}};
\node [image,right=of frame21] (frame22) 
    {\includegraphics[width=0.15\linewidth]{plots/population_eliteComparisons_c2_D=2_NON_DOM=2048_c=1.5.png}};
    \node[below = of frame22, node distance=0cm and 0cm] {\tiny{\textbf{Population Size}}};
\node[image,right=of frame22] (frame23)
    {\includegraphics[width=0.15\linewidth]{plots/population_eliteComparisons_c3_D=2_NON_DOM=2048_c=1.5.png}};
    \node[below = of frame23, node distance=0cm and 0cm] {\tiny{\textbf{Population Size}}};
\node[image,right=of frame23] (frame24)
    {\includegraphics[width=0.15\linewidth]{plots/population_eliteComparisons_c4_D=2_NON_DOM=2048_c=1.5.png}};
    \node[below = of frame24, node distance=0cm and 0cm] {\tiny{\textbf{Population Size}}};
\node[image,right=of frame24] (frame25)
    {\includegraphics[width=0.15\linewidth]{plots/population_eliteComparisons_c5_D=2_NON_DOM=2048_c=1.5.png}};
    \node[below = of frame25, node distance=0cm and 0cm] {\tiny{\textbf{Population Size}}};
\node[image,right=of frame25] (frame26)
    {\includegraphics[width=0.15\linewidth]{plots/population_eliteComparisons_c6_D=2_NON_DOM=2048_c=1.5.png}};
    \node[below = of frame26, node distance=0cm and 0cm] {\tiny{\textbf{Population Size}}};

\end{tikzpicture}
    \centering
    \includegraphics[height=0.25cm]{legend}
    \caption{\textbf{The ratio between the number of comparisons taken to update the elite set per edit operation. Results for population,} c = 1.5\textbf{. Top row:} Elite = 128 \textbf{. Middle row:} Elite = 512 \textbf{. Bottom row:} Elite = 2048\textbf{.}}
    \label{fig:elite_c=1.5}
\end{figure}

\subsubsection{Dimensions}
% introduce dimension graph
Figure \ref{fig:timings_dimensions} shows on a logarithmic scale how the number of comparisons required to update the elite set increases for populations with dimensions, $D$ $=$ ${2, 5, 10}.$ The value c is $1.0$ and the true size of the elite set is $512$. This time the objective values are not isotropic as being so would mean the nature of the dominance relations would not change. The number of comparisons has been chosen to be compared instead of timing as naturally the length of the objective vector has increased thus there is more to compare. It is of interest here to determine if the Pareto dominance relations between solutions can still lead to a well formed tree structure.

%%%%%%%%%%%%%
% Dimensions
%   Dimensions=2
%   Dimensions=5
%   Dimensions=10
%%%%%%%%%%%%%
\begin{figure}[h]
    \centering
    \begin{tikzpicture}[image/.style = {inner sep=0pt, outer sep=0pt}, node distance = 1mm and 1mm] 
% Dimensions = 2
\node [image] (frame1)
    {\includegraphics[width=0.15\linewidth]{plots/dimensions/population_comparisons_c1_D=2_NON_DOM=512_c=1.0.png}};
    \node[left = of frame1, node distance=0cm and 0cm, rotate=90, anchor=center] {\tiny{\textbf{Comparisons}}};
    \node[above = of frame1, node distance=0cm and 0cm] {$C_1$};
\node [image,right=of frame1] (frame2) 
    {\includegraphics[width=0.15\linewidth]{plots/dimensions/population_comparisons_c2_D=2_NON_DOM=512_c=1.0.png}};
    \node[above = of frame2, node distance=0cm and 0cm] {$C_2$};
\node[image,right=of frame2] (frame3)
    {\includegraphics[width=0.15\linewidth]{plots/dimensions/population_comparisons_c3_D=2_NON_DOM=512_c=1.0.png}};
    \node[above = of frame3, node distance=0cm and 0cm] {$C_3$};
\node[image,right=of frame3] (frame4)
    {\includegraphics[width=0.15\linewidth]{plots/dimensions/population_comparisons_c4_D=2_NON_DOM=512_c=1.0.png}};
    \node[above = of frame4, node distance=0cm and 0cm] {$C_4$};
\node[image,right=of frame4] (frame5)
    {\includegraphics[width=0.15\linewidth]{plots/dimensions/population_comparisons_c5_D=2_NON_DOM=512_c=1.0.png}};
    \node[above = of frame5, node distance=0cm and 0cm] {$C_5$};
\node[image,right=of frame5] (frame6)
    {\includegraphics[width=0.15\linewidth]{plots/dimensions/population_comparisons_c6_D=2_NON_DOM=512_c=1.0.png}};
    \node[above = of frame6, node distance=0cm and 0cm] {$C_6$};

% Dimensions = 5
\node [image] (frame11) [image,below = of frame1]
    {\includegraphics[width=0.15\linewidth]{plots/dimensions/population_comparisons_c1_D=5_NON_DOM=512_c=1.0.png}};
    \node[left = of frame11, node distance=0cm and 0cm, rotate=90, anchor=center] {\tiny{\textbf{Comparisons}}};
\node [image,right=of frame11] (frame12) 
    {\includegraphics[width=0.15\linewidth]{plots/dimensions/population_comparisons_c2_D=5_NON_DOM=512_c=1.0.png}};
\node[image,right=of frame12] (frame13)
    {\includegraphics[width=0.15\linewidth]{plots/dimensions/population_comparisons_c3_D=5_NON_DOM=512_c=1.0.png}};
\node[image,right=of frame13] (frame14)
    {\includegraphics[width=0.15\linewidth]{plots/dimensions/population_comparisons_c4_D=5_NON_DOM=512_c=1.0.png}};
\node[image,right=of frame14] (frame15)
    {\includegraphics[width=0.15\linewidth]{plots/dimensions/population_comparisons_c5_D=5_NON_DOM=512_c=1.0.png}};
\node[image,right=of frame15] (frame16)
    {\includegraphics[width=0.15\linewidth]{plots/dimensions/population_comparisons_c6_D=5_NON_DOM=512_c=1.0.png}};

% Dimensions = 10
\node [image] (frame21) [image,below = of frame11]
    {\includegraphics[width=0.15\linewidth]{plots/dimensions/population_comparisons_c1_D=10_NON_DOM=512_c=1.0.png}};
    \node[left = of frame21, node distance=0cm and 0cm, rotate=90, anchor=center] {\tiny{\textbf{Comparisons}}};
    \node[below = of frame21, node distance=0cm and 0cm] {\tiny{\textbf{Population Size}}};
\node [image,right=of frame21] (frame22) 
    {\includegraphics[width=0.15\linewidth]{plots/dimensions/population_comparisons_c2_D=10_NON_DOM=512_c=1.0.png}};
    \node[below = of frame22, node distance=0cm and 0cm] {\tiny{\textbf{Population Size}}};
\node[image,right=of frame22] (frame23)
    {\includegraphics[width=0.15\linewidth]{plots/dimensions/population_comparisons_c3_D=10_NON_DOM=512_c=1.0.png}};
    \node[below = of frame23, node distance=0cm and 0cm] {\tiny{\textbf{Population Size}}};
\node[image,right=of frame23] (frame24)
    {\includegraphics[width=0.15\linewidth]{plots/dimensions/population_comparisons_c4_D=10_NON_DOM=512_c=1.0.png}};
    \node[below = of frame24, node distance=0cm and 0cm] {\tiny{\textbf{Population Size}}};
\node[image,right=of frame24] (frame25)
    {\includegraphics[width=0.15\linewidth]{plots/dimensions/population_comparisons_c5_D=10_NON_DOM=512_c=1.0.png}};
    \node[below = of frame25, node distance=0cm and 0cm] {\tiny{\textbf{Population Size}}};
\node[image,right=of frame25] (frame26)
    {\includegraphics[width=0.15\linewidth]{plots/dimensions/population_comparisons_c6_D=10_NON_DOM=512_c=1.0.png}};
    \node[below = of frame26, node distance=0cm and 0cm] {\tiny{\textbf{Population Size}}};

\end{tikzpicture}
    \centering
    \includegraphics[height=0.25cm]{legend}
    \caption{\textbf{The cumulative time taken to update the data structure per operation. Top row:} D = 2 \textbf{. Middle row:} D = 5 \textbf{. Bottom row:} D = 10\textbf{.}}
    \label{fig:timings_dimensions}
\end{figure}

% discuss dimensions results
The number of comparisons taken is shown to increase when the number of dimensions is changed. The magnitude of the increase in comparisons differs between guardian assignment method however. It can be seen that $C_{1-4}$ are affected heavier by an increased number of objectives in comparison to $C_{5-6}$. This is likely because $C_{5-6}$ distribute children between potential guardians better thus no one guardian will have significantly more children than any other. The increase in comparisons holds true for each guardian assignment method therefore it can be concluded that the archive is susceptible to changes in the number of objectives. This occurs because guardian assignment is built upon the transitive nature of Pareto dominance relations, which have a tendency to break down in higher dimensions as solutions tend to mutually dominate each other. This leads to shorter trees, which in turn leads to an increased search space thus an increased number of comparisons. This is particularly expensive when a single solution has a large number of children as they each will need reassigning a guardian.

\subsubsection{Linear List Comparison}
The evaluation criteria specifiy that the archive should out perform a standard linear list approach therefore a basic linear list implementation has been provided to act as a benchmark. It has been written under the same \texttt{DynamicArchive} interface, conforming to the same set of unit tests to ensure quality. Here the number of comparisons taken has been compared to show the efficiency gains made by the \texttt{GuardianArchive}.
Figure \ref{fig:linear_timings} demonstrates how the \texttt{GuardianArchive} performs in comparison to the linear list approach. The elite archive has size 512 and non dominated solutions are distributed according to $c$ = 1.0. $C_1$ has been chosen as the guardian assignment method for its consistency across previous plots.

%%%%%%%%%%%%%
% Linear
%   Guardian c1
%   Guardian c4?
%   Linear
%%%%%%%%%%%%%
\begin{figure}[h]
    \centering
    \begin{tikzpicture}[image/.style = {inner sep=0pt, outer sep=0pt}, node distance = 1mm and 10mm]
% Guardian C1
\node [image] (frame1)
    {\includegraphics[width=0.15\linewidth]{plots/linear/population_comparisons_c1_D=2_NON_DOM=128_c=1.0.png}};
    \node[left = of frame1, node distance=0cm and 0cm, rotate=90, anchor=center] {\tiny{\textbf{Comparisons}}};
    \node[above = of frame1, node distance=0cm and 0cm] {Elite set = 128};
\node [image,right=of frame1] (frame2) 
    {\includegraphics[width=0.15\linewidth]{plots/linear/population_comparisons_c1_D=2_NON_DOM=512_c=1.0.png}};
    \node[above = of frame2, node distance=0cm and 0cm] {Elite set = 512};
\node[image,right=of frame2] (frame3)
    {\includegraphics[width=0.15\linewidth]{plots/linear/population_comparisons_c1_D=2_NON_DOM=2048_c=1.0.png}};
    \node[above = of frame3, node distance=0cm and 0cm] {Elite set = 2048};

% Guardian C4
\node [image] (frame11) [image,below = of frame1]
    {\includegraphics[width=0.15\linewidth]{plots/linear/population_comparisons_c4_D=2_NON_DOM=128_c=1.0.png}};
    \node[left = of frame11, node distance=0cm and 0cm, rotate=90, anchor=center] {\tiny{\textbf{Comparisons}}};
\node [image,right=of frame11] (frame12) 
    {\includegraphics[width=0.15\linewidth]{plots/linear/population_comparisons_c4_D=2_NON_DOM=512_c=1.0.png}};
\node[image,right=of frame12] (frame13)
    {\includegraphics[width=0.15\linewidth]{plots/linear/population_comparisons_c4_D=2_NON_DOM=2048_c=1.0.png}};
    
% Linear
\node [image] (frame21) [image,below = of frame11]
    {\includegraphics[width=0.15\linewidth]{plots/linear/linear_population_comparisons_c1_D=2_NON_DOM=128_c=1.0.png}};
    \node[left = of frame21, node distance=0cm and 0cm, rotate=90, anchor=center] {\tiny{\textbf{Comparisons}}};
\node [image,right=of frame21] (frame22) 
    {\includegraphics[width=0.15\linewidth]{plots/linear/linear_population_comparisons_c1_D=2_NON_DOM=512_c=1.0.png}};
\node[image,right=of frame22] (frame23)
    {\includegraphics[width=0.15\linewidth]{plots/linear/linear_population_comparisons_c1_D=2_NON_DOM=2048_c=1.0.png}};

\end{tikzpicture}
    \centering
    \includegraphics[height=0.25cm]{legend}
    \caption{\textbf{The cumulative time taken to update the data structure per operation. Top row:} D = 2 \textbf{. Middle row:} D = 5 \textbf{. Bottom row:} D = 10\textbf{.}}
    \label{fig:linear_timings}
\end{figure}

% linear analysis

\subsection{Results Discussion}
% what the archive does
% full average case complexity not possible, hence empirical only
% scheme performs better under different scenarios
% Parallelisation, lead into future work - prevalence of multi-core computational resources in modern computing infrastructure, the performance of parallelised versions of these structures is an area we are keen to explore.

This report has proposed a suggested framework for the managing of an archive that allows the quick discovery of the non-dominated elite set from a constantly evolving population. It is able to handle both when new solutions are added to the population and when the objective values of existing solutions change during the optimisation process which is of particular use in dynamic and noisy optimisation.\\
The archive's update mechanism relies on assigning a \textit{guardian dominator} to every non-elite solution, with such relations being used to reduce the number of domination comparisons required. 
The empirical results show that by assigning such guardian dominators, the location of the elite set in a constantly evolving population can be located quickly. Performance of the archive does very much depend on the way in which the population evolves with it being far better suited to scenarios where the elite set is fast moving. This is dependent on the guardian assignment method chosen however.\\
It is not determined whether the provided implementation's performance could further benefit from parallelisation, as multi-core architecture is prevalent in modern computing infrastructure, and utilising such has the potential to lead to significant performance improvements.

\section{Critical Assessment Evaluation}
This section outlines the scheme of work undertaken, offering a reflective look at how the project developed, justifying the decisions that were made and highlighting any errors that could be rectified through better planning and foresight.

\subsection{Literature}
%compare with literature
The investigation of the existing literature on the role of the data structure in a MOOP showed this was a condensed, mostly unexplored area of MOEA which had little to none agreed standardised approaches to how one should maintain an archive of solutions. The most common approach was to utilise a linear list which would take magnitudes more comparisons to update the elite set than a specialised data structure. It was found there were several approaches for storing bounded archives of elite static solutions, however optimisation problems which were dynamic in nature did not appear to have a specialised data structure besides the reference implementation \cite{Fieldsend2014}. The lack of existing literature in this precise area therefore meant that the linear list was the approach to beat, with any improvements seen to be significantly advantageous.\\
The literature review could of been improved by taking a more refined look at the data structures themselves, rather than the role they play. This would of allowed for a greater variety of data structures to be reviewed, particularly focusing on such structures that are either dynamic or non elitist as this would of provided greater guidance when developing the aims and requirements.

\subsection{Aims and Requirements}
% aims and requirements
The aims and functional requirements outlined in the project specification can now be evaluated for effectiveness.\\
The archive offers functionality to: evaluate new locations; change a dominated member; change a non-dominated member; retrieve the Pareto set at the current time step, and is unbounded. Therefore the core evaluation alone would conclude that the project has been a success in meeting its goal, but this does not reflect quality.
When evaluating the non-functional criteria, the quality of the archive can subsequently be inferred. From what was set out to be desired, the data structure conforms to a standardised interface, both for unbounded non-elitist dynamic archives, their dynamic solutions, and for the specific implementation of a \texttt{GuardianArchive}. It is able to sufficiently handle dynamic, robust, and noisy data by detecting duplicate decision and objective values, determined by a constant \texttt{EPS} value. The data structure is not unfortunately thread safe as this would of undermined the empirical analysis performed against it by allowing architecture to play a greater role in the timing characteristics. This would make comparison against other data structures complex if they too have not benefited from the dividends of parallelisation.\\ 
The desirable performance requirement was to outperform the basic linear list approach. The factor this would depend heavily on the guardian assignment technique used which was subsequently investigated by the empirical analysis where the different scenarios under which performance could be maximised were cross examined. Although no reference linear list implementation was provided, the results gathered do show great orders of magnitude in the reduction of the number of comparisons.

\subsection{Process and Approach}
%project process and approach
The process used to develop the implementation proved to be effective as the benefits of an agile approach were reaped. Having a backlog of tasks that could be updated as problems became more refined proved to be beneficial in keeping the project aimed towards the specific problem at hand, rather than being led astray by other problems which some of the literature had focused on. The approach could perhaps have been strengthened further by more rigours significant deadlines, as the frequency of a weekly meeting did not always convey the urgency that work would need to be completed by in order for it to remain equally distributed amongst sprints.
%improvements: plan efficiency improvements better, testing

\subsection{Fitness for Purpose}
% results summary, fitness for purpose
% performance / testing
The fitness for purpose of the provided library should be decided by the functionality, the results produced, and the performance given.\\
Firstly, the library provides all of the functional requirements expected of it, brought together under an interface. Secondly, the results produced were shown to be accurate, reliable and consistent between scenarios. The unit tests written are all passing without fail and the implementation has not thrown any exceptions hence the fitness for purpose is left to be evaluated by the empirical analysis.\\
The empirical analysis performed demonstrated that the archive is capable of performing numerous operations, under a reasonable rate of growth for both time taken, and the number of comparisons required. The results concluded that the archive is best suited for a fast moving elite set provided that the correct guardian assignment method is chosen. When comparing the performance to a linear list, it is clear that the number of comparisons taken to update the elite set is far fewer here. The extent of the analysis could of been greater with further guardian assignment techniques tested, as well as delving into dynamically changing the guardian assignment technique as the population grows. Therefore the fitness for purpose could be further identified in greater detail than has been done so here.

\subsection{Limitations}
% limitations
% no ordering of pareto set in terms of quality
% no sorting functionality
Despite the archives success in achieving its intended outcomes, there are some limitations to the functionality on offer. There is no ordering of the Pareto set itself which would be useful as an indicator of quality of solutions. Currently, it would be down to the user to refine the elite set in order of favourability as a solution to the MOOP. Solutions could be sorted by the number of children they have where the more solutions indicates better quality as that solution is likely to dominate more of the dominated set. This would be a simple addition to add to the interface, with the bulk of the work being the consultation of existing frameworks to determine the best format to add this functionality.

\subsection{Conclusion}
In summary, the results proved successful, fulfilling the research hypothesis, and hopefully inspiring a future avenue of work into unbounded, non-elitist dynamic archives. The tree structure proved an effective method of representing and organising dominance relations between multiple solutions as it was able to significantly reduce the number of comparisons taken to update the estimate the elite set as demonstrated in the empirical analysis.
It is hoped by providing a suitable data structure for a MOEA to use that the overall computation time to solve a MOOP will be significantly reduced. Although there are multiple alternative data structures, as demonstrated by the literature, use cases can vary dramatically and there is no one data structure fits all approach hence it is imperative that a thorough empirical analysis is performed to identify if, and when said data structure should be employed.

\subsection{Future Work}
% dynamically changing the population parameters
% order the Pareto set/children by fewest guarded
% thread safe
The prevalence of parallel architecture in modern computing infrastructure should be a strong motivating factor for developing a thread safe implementation of the guardian archive. Thread safety would allow for multiple solutions to be operated on under one time step, thus reducing the bottleneck of having a long queue of solutions to operate on. It would perhaps be beneficial to consider the ordering of solutions for insertion as the order would strongly affect the amount of computation required.\\
Besides the addition of thread safety, performance could further be enhanced by an ordering of the elite set, and possibly the children of each guardian. This would be beneficial when guardians are assigned based on which guards the fewest non-dominated children. This will incur the increased cost of maintaining an ordering therefore the trade off will have to be investigated - a multi objective optimisation problem in its own right.\\
Furthermore, a comprehensive analysis could be done on adapting the guardian assignment method, based on how the population is evolving. This would make the archive more resilient to differences between specific problems, thus giving it greater purpose as a general framework to use.

\newpage
\bibliography{report}
\bibliographystyle{IEEEtran}
\end{document} 

